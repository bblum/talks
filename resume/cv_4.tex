%%%%%%%%%%%%%%%%%%%%%%%%%%%%%%%%%%%%%%%%%
% Medium Length Professional CV
% LaTeX Template
% Version 2.0 (8/5/13)
%
% This template has been downloaded from:
% http://www.LaTeXTemplates.com
%
% Original author:
% Trey Hunner (http://www.treyhunner.com/)
%
% Important note:
% This template requires the resume.cls file to be in the same directory as the
% .tex file. The resume.cls file provides the resume style used for structuring the
% document.
%
%%%%%%%%%%%%%%%%%%%%%%%%%%%%%%%%%%%%%%%%%

%----------------------------------------------------------------------------------------
%	PACKAGES AND OTHER DOCUMENT CONFIGURATIONS
%----------------------------------------------------------------------------------------

\documentclass{resume} % Use the custom resume.cls style

\usepackage[left=0.6in,top=0.6in,right=0.6in,bottom=0.6in]{geometry} % Document margins
\usepackage[charter]{mathdesign}

\name{Ben Blum} % Your name
\address{5000 Forbes Ave \\ Gates-Hillman 9009 \\ Pittsburgh, PA 15213}
\address{\texttt{bblum@cs.cmu.edu} \\ \texttt{https://github.com/bblum}} % Your phone number and email
\address{6704 Thomas Blvd Apt 2L \\ Pittsburgh, PA 15208}

\begin{document}

%----------------------------------------------------------------------------------------
%	EDUCATION SECTION
%----------------------------------------------------------------------------------------

\begin{rSection}{Education}

{\bf Carnegie Mellon University} \hfill {\em June 2011} \\ 
B.S. in Computer Science

{\bf Carnegie Mellon University} \hfill {\em June 2012} \\ 
M.S. in Computer Science

{\bf Carnegie Mellon University} \hfill {\em 2018 (expected)} \\ 
Ph.D. in Computer Science


\end{rSection}

%----------------------------------------------------------------------------------------
%	WORK EXPERIENCE SECTION
%----------------------------------------------------------------------------------------

\begin{rSection}{Experience}

\begin{rSubsection}{Carnegie Mellon University}{Summer 2011 - Present}{Graduate Student}{Pittsburgh, PA}
\item Developed Landslide, a systematic concurrency testing framework for low-level userspace and kernel systems
\item Conducted educational user studies offering Landslide to students during operating systems class projects
\end{rSubsection}

\begin{rSubsection}{Mozilla}{Summers 2012 \& 2013}{Research Intern}{Mountain View, CA}
\item Helped develop the Rust language runtime (green-thread scheduler, stack management, concurrency primitives)
\item Developed an API for type-safe, race-free mutable shared state between threads using Rust's borrow checker
\item Contributed to the design of a sound type system for Rust's closures
\end{rSubsection}

\begin{rSubsection}{Google}{Summers 2009 \& 2010}{Software Engineering Intern}{Mountain View, CA}
\item Implemented atomic per-process migration algorithm for Linux kernel cgroups
\end{rSubsection}

\begin{rSubsection}{Carnegie Mellon University}{Spring 2009 - Fall 2012}{Teaching Assistant and Instructor}{Pittsburgh, PA}
\item TA for 15-213, Introduction to Computer Systems (3 semesters)
\item TA for 15-410, Operating System Design and Implementation (3 semesters)
\item Instructor for 98-172, Great Practical Ideas in Computer Science (1 semester)
\end{rSubsection}

\end{rSection}

%----------------------------------------------------------------------------------------
%	TECHNICAL STRENGTHS SECTION
%----------------------------------------------------------------------------------------

\begin{rSection}{Publications}

\textbf{Stateless Model Checking with Data-Race Preemption Points.} Ben Blum and Garth Gibson. OOPSLA 2016.
%ACM SIGPLAN International Conference on Object-Oriented Programming, Systems, Languages, and Applications (OOPSLA) 2016.

\textbf{Parrot: A Practical Runtime for Deterministic, Stable Threads.} Heming Cui et al. (5th author). SOSP 2013.
%ACM Symposium on Operating Systems Principles (SOSP) 2013.

\textbf{Soundness Proofs for Iterative Deepening.} Ben Blum. Tech report CMU-PDL-16-103.

\textbf{Landslide: Systematic Dynamic Race Detection in Kernel Space.} Ben Blum. MS thesis CMU-CS-12-118.

\end{rSection}

\begin{rSection}{Skills}

\begin{tabular}{ @{} >{\bfseries}l @{\hspace{6ex}} l }
	Languages (proficient) & C, Rust, Haskell, x86 assembly, Bash, LaTeX \\
	Languages (familiar) & C++, SML, Python, Lua \\
	Tools & git, vim, GNU toolchain, simics, bochs
\end{tabular}

\end{rSection}

%----------------------------------------------------------------------------------------
%	EXAMPLE SECTION
%----------------------------------------------------------------------------------------

%\begin{rSection}{Section Name}

%Section content\ldots

%\end{rSection}

%----------------------------------------------------------------------------------------

\end{document}
