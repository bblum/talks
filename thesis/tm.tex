\chapter{Transactions}
\label{chap:tm}

%\inspirationalquote{This paper presents earth-shaking work that deserves to be read by the entire computer science community.
%To that end, have you considered putting it in a thesis?}
%{Anonymous SIGBOVIK reviewer}

\inspirationalquote{
\begin{tabular}{p{0.47\textwidth}}
One can only build a sand castle where the sand is wet. \\
But where the sand is wet, the tide comes. \\
Yet we still build sand castles.
\end{tabular}}
{Yuri, Doki Doki Literature Club}


% notes to self abour marios tests
% counter: tests k threads on n iterations, benchmark xchg-aways vs txn. removed benchmarking, added assertion that no counts are lost.

% TODO

Transactional memory (TM) is a concurrency primitive
by which programmers may attempt an arbitrary sequence of shared memory accesses,
which will either be committed (made visible to other processors/threads) atomically,
such that no intermediate state modification is ever visible,
or in the case of a conflict which would prevent such,
discarded with an error code returned to allow the programmer to write a synchronized backup path.
%
Transactional memory specifications typically have three API functions, abstractly speaking:
\begin{itemize}
	\item  {\sf begin} begins a transaction,
		staging any subsequent shared memory accesses in some temporary thread-local storage,
		and checking for conflicts with the accesses of any other threads or CPUs.
		If the transaction is unsuccessful, as described below,
		{\sf begin} instead returns an error code
		indicating the programmer should fall back to some other, possibly slower, synchronization method.
	\item {\sf end} ends a transaction,
		attempting to commit all staged accesses to the shared memory atomically with respect to
		reads or writes from other concurrently-executing code.
		If any of those accesses conflict (i.e., read/write or write/write)
		with any other access to the same memory since the transaction started,
		they are instead discarded and execution state reverts to the {\sf begin} with an error code as described above.
	\item {\sf abort} explicitly aborts a transaction,
		regardless of any memory conflicts,
		discarding changes and reverting execution as described above.
		Some implementations allow an arbitrary abort code to be specified
		which will appear in {\sf begin}'s error code.
\end{itemize}

% do they track changes in pthread-style TLS? or?
{\bf Implementations.}
Software TM implementations (STM) typically function as a library,
tracking staged memory accesses in local memory,
and aborting whenever a conflict is detected between two transactions' tracked accesses \cite{stm-pldi06}.
Hardware TM implementations (HTM) use processor-level hardware support,
which stages changes in per-CPU cache lines,
and may abort for STM's reason above \cite{htm-experience},
or additionally whenever a conflict is detected between one transaction's traced access and {\em any} other memory access,
or whenever a conflict occurs on the same cache line, not necessarily the same address,
or in case of any system interrupt or cache overflow.

{\bf Terminology.}
The world of hardware transactional memory is home to several more confusing acronyms in addition to ``HTM''.
Transactional Synchronization Extensions (TSX)
refers to Intel's implementation of HTM on the Haswell microarchitecture
\cite{htm-haswell}.
Restricted Transactional Memory (RTM)
refers to the {\tt xbegin}, {\tt xend}, and {\tt xabort} subset of TSX instructions,
which of course correspond to {\sf begin}, {\tt end}, and {\tt abort} listed above,
and which GCC exposes as C/C++ intrinsices named {\tt \_xbegin()}, {\tt \_xend()}, and {\tt \_xabort()}
\cite{htm-gcc}.
Hardware Lock Elision (HLE)
refers to the {\tt xacquire} and {\tt xrelease} subset of TSX instructions,
which extend the traditional interface to offer a slightly higher-level way to access the CPU feature,
optimized for simplicity for locking-like synchronization patterns
% other possible things to cite here:
% https://lwn.net/Articles/534758/
% https://software.intel.com/en-us/node/683688
\cite{hardware-lock-elision}.
In this thesis I focus on RTM, the more general (i.e., expressive (i.e., bug-prone)) interface,
and among all these acronyms restrict myself to ``HTM''
(when referring to transactional memory as a concurrency primitive in the abstract)
and ``TSX''
(when referring to Intel's implementation and/or GCC's intrinsics interface).
The non-pedantic reader may treat these as interchangeable.

An example TSX program is shown in \sect{\ref{sec:overview-tm}}.

%%%%%%%%%%%%%%%%%%%%%%%%%%%%%%%%%%%%%%%%%%%%%%%%%%%%%%%%%%%%%%%%%%%%%%%%%%%%%%%%

\section{Concurrency Model}

% TODO
