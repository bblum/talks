\chapter{Landslide}
\inspirationalquote{Somewhere is the promise of an uncharted trail, with 700 branching limbs and 700 ways to fail.}
{ThouShaltNot, "Cardinal Directions"}

Landslide is a model checker implemented as a plug-in module for x86 full-system simulators.
The program to be tested runs in a simulated environment,
and Landslide uses its access to the simulator's internal state to inspect and manipulate the memory and thread scheduling of the program as it executes.
As of this thesis's writing, Landslide supports the use of two possible simulators:

\begin{itemize}
	\item {\bf Simics} \cite{simics}, a proprietary simulator licensed commercially by Wind River, used at CMU in 15-410 to run Pebbles thread libraries and kernels, and
	\item {\bf Bochs} \cite{bochs}, an open-source (LGPL) simulator used at the University of Chicago, Berkeley, and other schools to run Pintos kernels.
\end{itemize}

% TODO: update this linkerino (repository name)
% TODO: add some backup linkeroos
The Bochs port of Landslide is likewise open-source and available at \url{https://github.com/bblum/bochs}.
% TODO: refresh
The HEAD commit at the time of writing is 8984021.

This chapter will discuss Landslide's outer and inner workings for both aspiring users and aspiring developers in all gory detail. Strap in.

\section{User interface}

asdfadsf

\section{Architecture}

\subsection{Scheduler}
\subsection{Memory analysis}
\subsection{Machine state manipulation}
\subsection{State space traversal}
% time travel, explore, arbiter
\subsection{Debugging output}
% symtable stuff goes here
% stack tracing goes here
% foundabug goes here
\subsection{Pebbles-specific features}
\subsection{Pintos-specific features}
\subsection{Handy scripts}

\section{Algorithms of note}

\subsection{Kernel annotations}
\subsection{Bug identification}
\subsection{Data race analysis}
\subsection{Preemption point identification}
\subsection{State space estimation}
\subsection{Dynamic Partial Order Reduction}
\subsection{Heuristic loop, deadlock, and synchronization detection}
