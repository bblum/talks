\chapter{Landslide}
\inspirationalquote{Somewhere is the promise of an uncharted trail, with 700 branching limbs and 700 ways to fail.}
{ThouShaltNot, "Cardinal Directions"}

Landslide is a model checker implemented as a plug-in module for x86 full-system simulators.
The program to be tested runs in a simulated environment,
and Landslide uses its access to the simulator's internal state to inspect and manipulate the memory and thread scheduling of the program as it executes.
As of this thesis's writing, Landslide supports the use of two possible simulators:

\begin{itemize}
	\item {\bf Simics} \cite{simics}, a proprietary simulator licensed commercially by Wind River, used at CMU in 15-410 to run Pebbles thread libraries and kernels, and
	\item {\bf Bochs} \cite{bochs}, an open-source (LGPL) simulator used at the University of Chicago, Berkeley, and other schools to run Pintos kernels.
\end{itemize}

% TODO: update this linkerino (repository name)
% TODO: add some backup linkeroos
The Bochs port of Landslide is likewise open-source and available at \url{https://github.com/bblum/bochs}.
% TODO: refresh
The HEAD commit at the time of writing is 8984021.
The Simics port is available upon request.

This chapter will discuss Landslide's outer and inner workings for both aspiring users and aspiring developers in all gory detail. Strap in.

\section{User interface}

This section describes the features of Landslide the average student user should expect to interact with.
% TODO: make more specific section reference
Separate user guides also exist, described in Chapter~\ref{chap:410}.

\subsection{Setup}

Two setup scripts are provided, one for each supported kernel architecture: {\tt p2-setup.sh} and {\tt pintos-setup.sh}.
The user should supply the directory containing their project implementation.
The latter script also supports arguments specifying which of the Pintos projects to target.
For example:
\begin{itemize}
	\item {\tt ./p2-setup.sh /path/to/my/p2}
	\item {\tt ./pintos-setup.sh /path/to/my/threads} (2nd argument defaults to ``{\tt threads}'')
	\item {\tt ./pintos-setup.sh /path/to/my/userprog userprog}
\end{itemize}

These scripts accomplish the following setup tasks (among other trivialities):
\begin{itemize}
	\item Copy the user's code into {\tt pebsim/p2-basecode/} or {\tt pebsim/pintos/},
		which contain a pre-annotated Pebbles reference kernel binary or pre-annotated Pintos basecode, respectively.
	\item Build the code in its new location.
	\item Run the instrumentation script on the resulting binary to let Landslide know where all the important functions are
		(see \sect{\ref{sec:landslide-glue}}).
\end{itemize}

\subsection{Running Landslide through Quicksand}

The preferred method of invoking Landslide is through Quicksand, the Iterative Deepening wrapper program which has all of Chapter~\ref{chap:quicksand} to itself.
% TODO: add quicksand symlink
This is done via the {\tt ./landslide} script in the top-level directory, which:
\begin{itemize}
	\item Checks if the user needs to run {\tt *-setup.sh} again, in case their source code was more recently updated than the existing annotated build (a common mistake),
	\item Passes its arguments through to {\tt id/landslide-id}, the Quicksand binary,
		and
	\item (If during the student user study,) compresses the resulting log files,
		creates a snapshot tarball of them and the current version of the user's code,
		and sends it to me for nefarious research purposes.
\end{itemize}

\subsubsection{Command-line argments}

The following command line arguments are recommended for the average user.

\begin{itemize}
	\item {\tt -p PROGRAM}: the name of the test case to invoke
	\item {\tt -t TIME}: wall-clock time limit, in seconds; or suffixed with one of {\tt ydhms} for years, days, hours, minutes, or seconds respectively (default 1h)
	\item {\tt -c CPUS}: maximum number of Landslide instances to run in parallel (defaults to half the number of system CPUs)
	\item {\tt -i INTERVAL}: interval of time between printing progress reports (default 10s)
	\item {\tt -d TRACEDIR}: directory for resulting bug traces (default current directory)
	\item {\tt -v}: verbose mode (issues output for each executed interleaving by each instance of landslide, makes progress reports more detailed, etc)
	\item {\tt -l}: leave Landslide log files from completed state spaces even when no bug was found (deleted automatically by default)
	\item {\tt -h}: print help text and exit immediately
\end{itemize}

The following ``secret'' arguments also exist, primarily for my own use in running experiments or debugging.

\begin{itemize}
	\item {\tt -C}: enable ``control experiment'' mode, i.e., run only 1 instance of Landslide, with all (non-data-race) preemption points enabled in advance
		% TODO: put a section reference here
	\item {\tt -I}: enable Iterative Context Bounding (requires {\tt -C}, although future work may relax this restriction);
		this generally causes bugs to be found faster should they exist, but degrades completion time
	\item {\tt -0}: enable Preempt-Everywhere mode (see \sect{\ref{sec:quicksand-eval}}, requires {\tt -C})
	\item {\tt -H}: use Limited Happens-Before for data-race analysis (see \sect{\ref{sec:background-hb}})
	\item {\tt -V}: use vector-clock-based Pure Happens-Before for data-race analysis (see \sect{\ref{sec:background-hb}})
	\item {\tt -X}: support transactional memory (see Chapter~\ref{chap:tm})
		% TODO: make a more specific section reference
	\item {\tt -A}: support multiple abort codes during transaction failure (see Chapter~\ref{chap:tm});
		required for testing programs which behave differently under different abort circumstances,
		but impacts the state space size
	\item {\tt -P}: support Pintos architecture (enabled automatically when {\tt pintos-setup.sh} is run)
	\item {\tt -4}: support Pebbles architecture (enabled automatically when {\tt p2-setup.sh} is run)
	\item {\tt -e ETAFACTOR}: configure heuristic state space ETA deferring factor (described in detail in {\tt id/option.c})
	\item {\tt -E ETATHRESH}: configure heuristic threshold of state space progress for judging ETA stability (described in detail in {\tt id/option.c})
\end{itemize}

Quicksand will automatically generate configuration files and invoke Landslide according to the process described in the next section.

\subsection{Running Landslide directly}

Rather than letting Quicksand juggle multiple instances of Landslide,
the user may run a single instance directly, optionally configuring the preemption points by hand.
This is recommended only for enthusiastic users who are annotating their own kernel.

The script {\tt pebsim/landslide} invokes Landslide thus.
It should be run from within the {\tt pebsim/} directory.
When supplied no arguments, it reads configuration options from {\tt pebsim/config.landslide}
(a bash script expected to define certain variables as described in \sect{\ref{sec:landslide-glue}}).
The user may optionally specify a file containing additional config directives
% , such as custom preemption points,
as an argument\footnote{
Quicksand actually supplies two such files as arguments: one ``static'' config file and one ''dynamic'' config file.
The former contains options which require recompiling Landslide (e.g., whether or not to use ICB is controlled by an {\tt \#ifdef} in Landslide's code),
while the latter contains options which Landslide interprets at runtime (e.g., which preemption points to use).
The static options do not change between Landslide instances in a single Quicksand run,
avoiding long Landslide start-up times.
	}.
Such supported options are as follows.

\subsubsection{Dynamic configuration options}
First, the following options may be changed without triggering a recompile of Landslide.

\begin{itemize}
	\item {\tt within\_function} % TODO
	\item {\tt within\_user\_function} - as above but finds the function in the userspace test program rather than the kernel code.
	\item {\tt without\_function} % TODO
	\item {\tt without\_user\_function} - difference to one above same as stated two above.
	\item {\tt data\_race ADDR TID LAST\_CALL CURRENT\_SYSCALL} - specifies a data-race preemption point.
		\begin{itemize}
			\item {\tt ADDR} shall be the code address (in hex) of the racing address,
			{\em before} the execution of which a preemption will be issued.
			\item {\tt TID} indicates a thread ID required to be running for this data race.
				To specify data race PPs across all threads at once, set {\tt FILTER\_DRS\_BY\_TID=0} (see next section).
			\item {\tt LAST\_CALL} indicates a code address required to be the site of the last {\tt call} instruction executed
				(similar to specifying a stack trace, but using a full stack trace here degrades performance too much),
				or 0 to not use this feature.
				From personal experience I found this option rather useless and recommend always supplying 0.
				% TODO: fix this section ref
				For further discussion see \sect{\ref{sec:quicksand-pps}}.
			\item {\tt CURRENT\_SYSCALL} indicates the system call number if a user-space data race comes from within a kernel system call which accesses user memory (Pebbles only).
				Usually 0 (i.e., not in kernel code) but {\tt deschedule}'s system call number is common as well.
		\end{itemize}
	\item {\tt input\_pipe FILENAME} - FIFO file used for receiving messages from Quicksand (e.g. to suspend or resume execution).
		Requires {\tt id\_magic} option to be set (next section below).
		The odds a human user will find success using this option by hand are infinitesimal.
	\item {\tt output\_pipe FILENAME} - as above but for sending messages.
\end{itemize}

\subsubsection{Static configuration options}

Next, configuration options which affect an {\tt \#ifdef} in Landslide and will trigger a recompile upon changing.
Unless otherwise specified these are boolean flags (1 or 0) and the example value shown indicates the default used if unspecified.

\begin{itemize}
	\item {\tt FILTER\_DRS\_BY\_TID=1} - configures whether to use the {\tt TID} parameter of {\tt data\_race} described above.
	\item % TODO moar
\end{itemize}

\subsection{Bug reports}

\section{Architecture}

\subsection{Scheduler}
\subsection{Memory analysis}
\subsection{Machine state manipulation}
\subsection{State space traversal}
% time travel, explore, arbiter
\subsection{Debugging output}
% symtable stuff goes here
% stack tracing goes here
% foundabug goes here
\subsection{Pebbles-specific features}
\subsection{Pintos-specific features}
\subsection{Handy scripts}
\label{sec:landslide-glue}

\section{Algorithms of note}

\subsection{Kernel annotations}
\subsection{Bug identification}
\subsection{Data race analysis}
\subsection{Preemption point identification}
\subsection{State space estimation}
\subsection{Dynamic Partial Order Reduction}
\subsection{Heuristic loop, deadlock, and synchronization detection}
