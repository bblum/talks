\chapter{Landslide}
\inspirationalquote{Somewhere is the promise of an uncharted trail, with 700 branching limbs and 700 ways to fail.}
{ThouShaltNot, "Cardinal Directions"}

Landslide is a model checker implemented as a plug-in module for x86 full-system simulators.
The program to be tested runs in a simulated environment,
and Landslide uses its access to the simulator's internal state to inspect and manipulate the memory and thread scheduling of the program as it executes.
As of this thesis's writing, Landslide supports the use of two possible simulators:

\begin{itemize}
	\item {\bf Simics} \cite{simics}, a proprietary simulator licensed commercially by Wind River, used at CMU in 15-410 to run Pebbles thread libraries and kernels, and
	\item {\bf Bochs} \cite{bochs}, an open-source (LGPL) simulator used at the University of Chicago, Berkeley, and other schools to run Pintos kernels.
\end{itemize}

% TODO: update this linkerino (repository name)
% TODO: add some backup linkeroos
The Bochs port of Landslide is likewise open-source and available at \url{https://github.com/bblum/bochs}.
% TODO: refresh
The HEAD commit at the time of writing is 8984021.
The Simics port is available upon request.

This chapter will discuss Landslide's outer and inner workings in all their gory detail.
It is intended for the aspiring developer or the ambitious user
and hence unlike other chapters is written in the style of documentation rather than as a report of research results.

\section{User interface}

This section describes the features of Landslide the average student user should expect to interact with.
% TODO: make more specific section reference
Separate user guides also exist, described in Chapter~\ref{chap:410}.

\subsection{Setup}

Two setup scripts are provided, one for each supported kernel architecture: {\tt p2-setup.sh} and {\tt pintos-setup.sh}.
The user should supply the directory containing their project implementation.
The latter script also supports arguments specifying which of the Pintos projects to target.
For example:
\begin{itemize}
	\item {\tt ./p2-setup.sh /path/to/my/p2}
	\item {\tt ./pintos-setup.sh /path/to/my/threads} (2nd argument defaults to ``{\tt threads}'')
	\item {\tt ./pintos-setup.sh /path/to/my/userprog userprog}
\end{itemize}

These scripts accomplish the following setup tasks (among other trivialities):
\begin{itemize}
	\item Copy the user's code into {\tt pebsim/p2-basecode/} or {\tt pebsim/pintos/},
		which contain a pre-annotated Pebbles reference kernel binary or pre-annotated Pintos basecode, respectively.
	\item Build the code in its new location.
	\item Run the instrumentation script on the resulting binary to let Landslide know where all the important functions are
		(see \sect{\ref{sec:landslide-glue}}).
\end{itemize}

\subsection{Running Landslide through Quicksand}
\label{sec:landslide-quicksand-options}

The preferred method of invoking Landslide is through Quicksand, the Iterative Deepening wrapper program which has all of Chapter~\ref{chap:quicksand} to itself.
% TODO: add quicksand symlink
This is done via the {\tt ./landslide} script in the top-level directory, which:
\begin{itemize}
	\item Checks if the user needs to run {\tt *-setup.sh} again, in case their source code was more recently updated than the existing annotated build (a common mistake),
	\item Passes its arguments through to {\tt id/landslide-id}, the Quicksand binary,
		and
	\item (If during the student user study,) compresses the resulting log files,
		creates a snapshot tarball of them and the current version of the user's code,
		and sends it to me for nefarious research purposes.
\end{itemize}

\subsubsection{Command-line argments}

The following command line arguments are recommended for the common user.

\begin{itemize}
	\item {\tt -p PROGRAM}: the name of the test case to invoke
	\item {\tt -t TIME}: wall-clock time limit, in seconds; or suffixed with one of {\tt ydhms} for years, days, hours, minutes, or seconds respectively (default 1h)
	\item {\tt -c CPUS}: maximum number of Landslide instances to run in parallel (defaults to half the number of system CPUs)
	\item {\tt -i INTERVAL}: interval of time between printing progress reports (default 10s)
	\item {\tt -d TRACEDIR}: directory for resulting bug traces (default current directory)
	\item {\tt -v}: verbose mode (issues output for each executed interleaving by each instance of landslide, makes progress reports more detailed, etc)
	\item {\tt -l}: leave Landslide log files from completed state spaces even when no bug was found (deleted automatically by default)
	\item {\tt -h}: print help text and exit immediately
\end{itemize}

The following ``secret'' arguments also exist, primarily for my own use in running experiments or debugging.

\begin{itemize}
	\item {\tt -C}: enable ``control experiment'' mode, i.e., run only 1 instance of Landslide, with all (non-data-race) preemption points enabled in advance
		% TODO: put a section reference here
	\item {\tt -I}: enable Iterative Context Bounding (requires {\tt -C}, although future work may relax this restriction);
		this generally causes bugs to be found faster should they exist, but degrades completion time
		(\sect{\ref{sec:landslide-icb}})
	\item {\tt -0}: enable Preempt-Everywhere mode (\sect{\ref{sec:quicksand-eval}}, requires {\tt -C})
	\item {\tt -H}: use Limited Happens-Before for data-race analysis (\sect{\ref{sec:background-hb}})
		(default for Pebbles kernelspace mode)
	\item {\tt -V}: use vector-clock-based Pure Happens-Before for data-race analysis (\sect{\ref{sec:background-hb}})
		(default for P2 userspace and Pintos modes)
	\item {\tt -X}: support transactional memory (Chapter~\ref{chap:tm})
		% TODO: make a more specific section reference
	\item {\tt -A}: support multiple abort codes during transaction failure (Chapter~\ref{chap:tm});
		required for testing programs which behave differently under different abort circumstances,
		but impacts the state space size
	\item {\tt -P}: support Pintos architecture (enabled automatically when {\tt pintos-setup.sh} is run)
	\item {\tt -4}: support Pebbles architecture (enabled automatically when {\tt p2-setup.sh} is run)
		% TODO: put a section reference
	\item {\tt -e ETAFACTOR}: configure heuristic state space ETA deferring factor (described in detail in {\tt id/option.c})
	\item {\tt -E ETATHRESH}: configure heuristic threshold of state space progress for judging ETA stability (described in detail in {\tt id/option.c})
\end{itemize}

Quicksand will automatically generate configuration files and invoke Landslide according to the process described in the next section.

\subsection{Running Landslide directly}
\label{sec:landslide-directly}

Rather than letting Quicksand juggle multiple instances of Landslide,
the user may run a single instance directly, optionally configuring the preemption points by hand.
This is recommended only for enthusiastic users who are annotating their own kernel.

The script {\tt pebsim/landslide} invokes Landslide thus.
It should be run from within the {\tt pebsim/} directory.
When supplied no arguments, it reads configuration options from {\tt pebsim/config.landslide}
(a bash script expected to define certain variables as described in \sect{\ref{sec:landslide-glue}}).
The user may optionally specify a file containing additional config directives
% , such as custom preemption points,
as an argument.\footnote{
Quicksand actually supplies two such files as arguments: one ``static'' config file and one ''dynamic'' config file.
The former contains options which require recompiling Landslide (e.g., whether or not to use ICB is controlled by an {\tt \#ifdef} in Landslide's code),
while the latter contains options which Landslide interprets at runtime (e.g., which preemption points to use).
The static options do not change between Landslide instances in a single Quicksand run,
avoiding long Landslide start-up times.
}
Such supported options are as follows.

\subsubsection{Dynamic configuration options}
\label{sec:landslide-dynamicconfig}

First, the following options may be changed without triggering a recompile of Landslide.
They are implemented as bash functions defined in {\tt pebsim/build.sh}.

\begin{itemize}
	\item {\tt within\_function FUNC} - adds {\tt FUNC} to a whitelist of functions required to appear in the current stack trace before identifying a preemption point (see \sect{\ref{sec:landslide-pps}})
	\item {\tt without\_function FUNC} - as above, but a blacklist instead of a whitelist
	\item {\tt within\_user\_function FUNC} - as two above but finds the function in the userspace test program rather than the kernel code.
	\item {\tt without\_user\_function FUNC} - difference to two above same as stated one above.
	\item {\tt data\_race ADDR TID LAST\_CALL CURRENT\_SYSCALL} - specifies a data-race preemption point.
		\begin{itemize}
			\item {\tt ADDR} shall be the code address (in hex) of the racing address,
			{\em before} the execution of which a preemption will be issued.
			\item {\tt TID} indicates a thread ID required to be running for this data race.
				To specify data race PPs across all threads at once, set {\tt FILTER\_DRS\_BY\_TID=0} (see next section).
			\item {\tt LAST\_CALL} indicates a code address required to be the site of the last {\tt call} instruction executed
				(similar to specifying a stack trace, but using a full stack trace here degrades performance too much),
				or 0 to not use this feature.
				From personal experience I found this option rather useless and recommend always supplying 0.
				% TODO: fix this section ref
				For further discussion see \sect{\ref{sec:quicksand-pps}}.
			\item {\tt CURRENT\_SYSCALL} indicates the system call number if a user-space data race comes from within a kernel system call which accesses user memory (Pebbles only).
				Usually 0 (i.e., not in kernel code) but {\tt deschedule}'s system call number is common as well.
		\end{itemize}
	\item {\tt input\_pipe FILENAME} - FIFO file used for receiving messages from Quicksand (e.g. to suspend or resume execution).
		Requires {\tt id\_magic} option to be set (next section below).
		The odds that a human user will find spiritual enlightenment through using this option by hand are infinitesimal.
	\item {\tt output\_pipe FILENAME} - as above but for sending messages.
\end{itemize}

\subsubsection{Static configuration options}
\label{sec:landslide-staticconfig}

Next, configuration options which affect an {\tt \#ifdef} in Landslide and will trigger a recompile upon changing.
%These span a wide variety of features, sorted below in subsections by roughly how interesting I think they are.
Unless otherwise specified these are boolean flags (1 or 0) and the example value shown indicates the default used if unspecified.

\begin{itemize}
\item {\bf Search algorithm options}
\begin{itemize}
	\item {\tt ICB=0} - enable Iterative Context Bounding (\sect{\ref{sec:landslide-icb}});
		corresponds to {\tt -I} in \sect{\ref{sec:landslide-quicksand-options}}.
	\item {\tt PREEMPT\_EVERYWHERE=0} - enable Preempt-Everywhere mode (\sect{\ref{sec:quicksand-eval}});
		corresponds to {\tt -0} in \sect{\ref{sec:landslide-quicksand-options}}.
	\item {\tt EXPLORE\_BACKWARDS=0} - configure whether, at each newly encountered preemption point,
		to allow the current thread to run first then later upon backtracking to preempt (0),
		or to issue preemptions first and then try continuing the current thread later (1).
		0 tends to produce shorter preemption traces while 1 tends to find bugs faster (\cite{landslide} \S{}8.7.1).
		Not compatible with ICB.
\end{itemize}

\item {\bf Memory analysis options}
\begin{itemize}
	\item {\tt PURE\_HAPPENS\_BEFORE=1} - select Pure Happens-Before (1) or Limited Happens-Before (2) (\sect{\ref{sec:background-hb}});
		corresponds to {\tt -V}/{\tt -H} in \sect{\ref{sec:landslide-quicksand-options}}.
	\item {\tt FILTER\_DRS\_BY\_TID=1} - configures whether to use the {\tt TID} parameter of {\tt data\_\allowbreak{}race} described above.
	\item {\tt FILTER\_DRS\_BY\_LAST\_CALL=0} - configures whether to use the {\tt LAST\_CALL} parameter of {\tt data\_race} described above.
	\item {\tt ALLOW\_LOCK\_HANDOFF=0} - configures lockset tracking to permit or disallow a lock taken by one thread to be released by another thread.%
		\footnote{If enabled, accesses performed by the second thread before unlocking will not be considered protected by that lock,
			as Landslide cannot infer what prior event abstractly represented the lock's ownership changing,
			leading to spurious data race reports.
			This could be solved in future work with a new annotation.}
	\item {\tt ALLOW\_REENTRANT\_MALLOC\_FREE=0} - allow two threads to be in {\tt malloc}, {\tt free}, or so on simultaneously without declaring it a bug.%
		\footnote{Used in Pintos, where those functions lock/unlock the heap mutex themselves rather than relying on a wrapper function to do so before invoking them.}
	\item {\tt TESTING\_MUTEXES=0} - configure ``mutex testing'' mode (1),
		in which the data race analysis will not consider a mutex's implementation to be protected by the mutex itself.
		In other words, the mutex's internal memory accesses will be flagged as data races,
		thereby enabling Landslide to verify the mutual exclusion property.
		Normally (0), Landslide assumes mutual exclusion is provided in order to efficiently find data races in the rest of the code.
		Quicksand will automatically set this option for P2s when {\tt -t mutex\_test} is specified.
\end{itemize}

\item {\bf Interface options}
\begin{itemize}
	\item {\tt TEST\_CASE=NAME} - configure the name of the test program to run (mandatory; no default)
	\item {\tt VERBOSE=0} - enable more verbose output
	\item {\tt BREAK\_ON\_BUG=0} - configure whether to exit the simulator or drop into a debug prompt when a bug is found. Simics only and not compatible with Quicksand.
	\item {\tt DONT\_EXPLORE=0} - if enabled, Landslide will not perform stateless model checking but rather will execute the default thread interleaving then exit (useful for manual inspection of preemption points).
	\item {\tt PRINT\_DATA\_RACES=0} - as it says on the tin (for stand-alone use; will message them to Quicksand regardless).
	\item {\tt TABULAR\_TRACE=1} - configure whether to emit bug reports to the console (0) or to an HTML trace file (1)
\end{itemize}
\end{itemize}

\subsection{Bug reports}

When Landslide finds a bug, it produces an execution trace of the particular interleaving of threads that led to the bug.
This takes the form of a two-dimensional table,
with a column for each thread,
and each row representing the continuous execution of one thread between two (not necessarily consecutive) preemption points.
In each row, the cell in the column corresponding to the executed thread will contain a stack trace,
indicating the code location of the preemption point {\em at the end} of that thread transition
(i.e., each stack trace indicates ``this thread ran until it reached the indicated line of code'').
The bug reports are formatted in html, recommended to be viewed in a web browser.
An example is shown in Figure~\ref{fig:bugreport}.

In addition to the preemption trace, the bug report provides some additional helpful information:
a stack trace of the current thread at the ultimate point when the bug was executed,
a message indicating the nature of the bug encountered,
statistics about the size of the state space,
and optionally additional information about the bug.%
\footnote{For certain types of bugs, not pictured here;
for example, use-after-frees will report separate stack traces
indicating when the corresponding heap block was last allocated and freed.
The intrepid source-diver may find all such cases of extra bug details
by searching for the macro {\tt FOUND\_A\_BUG\_HTML\_INFO} in Landslide's code.}


% TODO: check figure placement
\begin{figure}[t]
	\begin{center}
		% nb. generated from landslide-trace-1506524837.14.html
		\includegraphics[width=\textwidth]{bugreport.pdf}
	\end{center}
	\caption{Example preemption trace bug report.}
	\label{fig:bugreport}
\end{figure}

\section{Kernel annotations}

The educational experiments in this thesis focus on projects which students implement on top of provided kernel basecode which Landslide already ``understands''.
Such understanding is conferred via the annotations described in this section.
For P2 and Pintos students I supply these annotations behind the scenes,
but a CMU 15-410 student who wishes to use Landslide on their kernel project shall need to brave forth hereupon.

\subsection{config.landslide annotations}

The following annotations are specified in {\tt pebsim/config.landslide} akin to the static configuration options
described in \sect{\ref{sec:landslide-directly}}..
These specify the names of kernel functions, global variables, default values, and so on
which are required to accurately track the kernel's scheduler state:
{\tt CONTEXT\_SWITCH},
{\tt EXEC},
{\tt FIRST\_TID},
{\tt IDLE\_TID},
{\tt INIT\_TID},
{\tt MEMSET},
{\tt PAGE\_FAULT\_WRAPPER},
{\tt READLINE},
{\tt SFREE},
{\tt SHELL\_TID},
{\tt SPURIOUS\_INTERRUPT\_WRAPPER},
\\
{\tt THREAD\_KILLED\_ARG\_VAL},
{\tt THREAD\_KILLED\_FUNC},
{\tt TIMER\_WRAPPER},
{\tt VM\_USER\_COPY},
\\
{\tt VM\_USER\_COPY\_TAIL},
{\tt YIELD}.

Following are the less self-explanatory options.
\begin{itemize}
	\item {\tt PINTOS\_KERNEL=0} - configure Landslide for Pebbles (0) or Pintos (1) kernel architecture. Normally set automatically by the setup scripts.
	\item {\tt TESTING\_USERSPACE=1} - configure Landslide whether to test (i.e., focus preemption points, memory analysis, etc. on) the userspace or kernelspace code.
	\item {\tt CURRENT\_THREAD\_LIVES\_ON\_RQ=0} - Landslide infers the list of runnable threads from the {\tt tell\_landslide\_on\_rq()} and {\tt off\_rq()} annotations (described below).
		Some kernels\footnote{most, actually} remove the current thread from their runqueue,
		such that the abstract set of all runnable threads is actually the runqueue plus the current thread rather than just the runqueue.
		Other kernels\footnote{the author's own student kernel from long ago}
		leave the current thread on the runqueue,
		removing it only when it's descheduling and should actually be considered blocked.
		Set this option to 0 to support the former kernel type or 1 to support the latter.%
		\footnote{This option replaces the deprecated {\tt kern\_current\_extra\_runnable()} annotation from {\tt student.c} described in \cite{landslide} \S{}6.2.3.}
	\item {\tt PREEMPT\_ENABLE\_FLAG=NAME} - name of a global variable which the kernel uses to toggle scheduler preemptability, for kernels which may disable preemption without disabling interrupts.
		For kernels wherein preemptability is corresponds directly by interrupts, leave this option unspecified.
	\item {\tt PREEMPT\_ENABLE\_VALUE=VAL} - value of the above variable when preemption is enabled
		(usually 0; note that many kernels use a nesting depth counter where any positive value corresponds to disabled).%
		\footnote{These two options replace the deprecated {\tt kern\_ready\_for\_timer\_interrupt()} annotation from {\tt student.c} described in \cite{landslide} \${}6.2.3.}
	\item {\tt PATHOS\_SYSCALL\_IRET\_DISTANCE=VALUE} - indicate how much stack space is used by the reference kernel's system call wrappers.
		Used for cross-kernel-to-userspace stack traces;
		if unset, stack traces from kernel space will end at the system call boundary.
	\item {\tt PDE\_PTE\_POISON=VALUE} - indicate a poison value used in the page tables to indicate absent VM mappings to check for as well as checking the present bit (if unspecified, will check present bit only)
	\item {\tt BUG\_ON\_THREADS\_WEDGED=1} - set to 0 to disable deadlock detection but instead let the kernel keep receiving system interrupts when all threads appear blocked.%
		\footnote{once used in the bad old days; now recommended for debugging use only}
	\item {\tt TIMER\_WRAPPER\_DISPATCH=NAME} - used to manually indicate a label before the end of the timer interrupt assembly wrapper, in case the {\tt iret} instruction couldn't be found automatically (see {\tt pebsim/definegen.sh}).
	\item {\tt starting\_threads TID STARTS\_ON\_RQ} - specifies a system thread which already exists at the time {\tt tell\_landslide\_sched\_init\_done()} (see below) is called; {\tt TID} is the thread's ID and {\tt STARTS\_ON\_RQ} is 0 or 1 to indicate whether or not it starts on the system runqueue.
		Typical threads to use this for are init and idle.
	\item {\tt ignore\_sym NAME SIZE} - specifies a global variable {\tt NAME} of a given {\tt SIZE} in bytes whose memory accesses should be ignored for the purposes of DPOR and data race analysis.
		Typical symbols to use this for are the console or heap mutex.
	\item {\tt sched\_func NAME} - specifies a function whose memory accesses should all be ignored for the purposes of DPOR and data race analysis.
		Typical functions to use this for are the timer handler and context switcher.
	\item {\tt disk\_io\_func NAME} - specifies a function which may block a thread waiting for disk I/O (or other external interrupt) rather than blocking on another thread.
		If any threads are blocked in a disk I/O function during an apparent deadlock,
		Landslide will allow the kernel to idle until the simulator delivers the appropriate interrupt,
		rather than declaring a bug.
	\item {\tt ignore\_dr\_function NAME USERSPACE} - specifies a function whose memory access should not be counted as data races (but still be considered memory conflicts for DPOR).
		{\tt USERSPACE} should be 0 or 1 to denote a kernel-space or user-space function respectively.
\end{itemize}

\subsection{In-kernel code annotations}

The following annotations are provided as C functions
which a kernel author shall include in their source code and call at appropriate times.
The functions' actual implementations are empty;
rather they serve as labels whose positions the annotation scripts extract
along with the other various annotations from the previous section.
Some of these are mandatory for Landslide to function properly,
while others serve to improve or otherwise manipulate the state space.

\subsubsection{Mandatory annotations}

\begin{itemize}
\item {\tt tell\_landslide\_thread\_switch(int new\_tid)} - to be called during context switch, indicating the newly-running thread
	(must be called with interrupts and/or scheduler preemption disabled)
\item {\tt tell\_landslide\_sched\_init\_done()} - to be called after scheduler initialization,
	indicating the point after which Landslide should begin analysis.
	Any threads already initialized before this point (init, idle, etc) should be specified with {\tt starting\_threads} (previous subsection).
\item {\tt tell\_landslide\_forking()} - to be called whenever a new thread is created,
	``immediately'' before the next {\tt thread\_switch()} or {\tt on\_rq()} call for that new thread (i.e., this call sets a flag which the next instance of either of the latter will check to see if the indicated thread is new).
	Most Pebbles kernels will call this twice; once in {\tt fork} and once in {\tt thread\_fork}.
\item {\tt tell\_landslide\_vanishing()} - to be called whenever a thread ceases to exist,
	``immediately'' before the next {\tt thread\_switch()} or {\tt off\_rq()} call for the exiting thread
	(works similarly to above).
\item {\tt tell\_landslide\_sleeping()} - to be called whenever a thread is about to {\tt sleep()} waiting for timer interrupts,
	``immediately'' before the next {\tt thread\_switch()} or {\tt off\_rq()} call for the sleeping thread
	(similar to the above).
	Landslide considers sleeping threads to be runnable as normal (they will just take more timer interrupts to arrive at),
	so this call is necessary to distinguish from the case when a thread is descheduled on a non-timer event.
\item {\tt tell\_landslide\_thread\_on\_rq(int tid)} - to be called when a thread is added to the runqueue
	(must be called with interrupts and/or scheduler preemption disabled).
\item {\tt tell\_landslide\_thread\_off\_rq(int tid)} - dual of the above.
	If {\tt CURRENT\_THREAD\_\allowbreak{}LIVES\_ON\_RQ=0} (described above), this should be invoked (among other times) during context switch with the TID of the thread about to start running.
	Alternatively (thanks sully), even for a kernel which takes the current thread off its literal runqueue,
	the annotator may use these two calls to indicate the ``abstract runqueue'' which includes the current thread as well,
	and set {\tt CURRENT\_THREAD\_LIVES\_ON\_RQ=1}.
\end{itemize}

\subsubsection{Optional annotations}

\begin{itemize}
\item {\tt tell\_landslide\_preempt()} - specifies a preemption point.
	Subject to the constraints of {\tt within\_function}/{\tt without\_function};
	hence may be ignored if used with Quicksand.
\item {\tt tell\_landslide\_dump\_stack()} - instructs Landslide to print a stack trace whenever this point is reached (for debugging purposes).
\end{itemize}

\subsubsection{Optional but strongly recommended annotations}

The following annotations enable Landslide to track locksets for data race analysis.
If not provided, it will be as if Landslide assumes no guarantees about mutual exclusion or happens-before,
and hence will identify all memory conflicts as data races.
(Note that the corresponding instrumentation for P2s is achieved automatically,
as the names of the mutex interface are mandated by the project specification.)

\begin{itemize}
\item {\tt tell\_landslide\_mutex\_locking(void *mutex\_addr)} - indicates the beginning of the lock routine for
	whatever synchronization API Landslide should treat as the primitive for data race detection.
	In Pintos this is the {\tt sema\_*()} function family; in Pebbles they may be called anything.
\item {\tt tell\_landslide\_mutex\_blocking(int owner\_tid)} - called ``immediately'' before a thread becomes blocked on the mutex.
	Definition of ``immediately'' similar to the {\tt forking()} and friends annotations above.
	{\tt owner\_tid} allows Landslide to efficiently unblock/re-block threads when the mutex holder changes
	(rather than relying on heuristic yield-loop detection);
	see {\tt kern\_mutex\_block\_others()} and {\tt deadlocked()} in {\tt schedule.c} for implementation details.
\item {\tt tell\_landslide\_mutex\_locking\_done(void *mutex\_addr)} - indicates the end of the lock routine.
\item {\tt tell\_landslide\_mutex\_trylocking(void *mutex\_addr)} - indicates the beginning of the trylock routine (if present).
\item {\tt tell\_landslide\_mutex\_trylocking\_done(void *mutex\_addr, int succeeded)} -
	indicates when a thread is finished trylocking, even if it failed to get the lock (indicated by {\tt succeeded}).
\item {\tt tell\_landslide\_mutex\_unlocking(void *mutex\_addr)} - indicates the beginning of the unlock routine.
\item {\tt tell\_landslide\_mutex\_unlocking\_done()} - indicates the end of the unlock routine.
\end{itemize}

\section{Architecture}

This section documents the organization of code within Landslide.
Unless otherwise specified, Landslide's code lives in
%C files live in
{\tt work/modules/landslide/} (Simics implemenation) or {\tt src/bochs-2.6.8/instrument/landslide/} (Bochs implementation)
relative to the repository root.

\subsection{Scheduler}
\subsection{Memory analysis}
\subsection{Machine state manipulation}
\subsection{State space traversal}
% time travel, explore, arbiter
\subsection{Bug-finding output}
% symtable stuff goes here
% stack tracing goes here
% foundabug goes here
\subsection{Pebbles-specific features}
\subsection{Pintos-specific features}
\subsection{Handy scripts}
\label{sec:landslide-glue}

The options specified in \sect{\ref{sec:landslide-directly}}
are handled by a family of gross shell scripts that live in {\tt pebsim/}.

\begin{itemize}
	\item {\tt landslide} is the outermost script invoked by Quicksand (or by a \sect{\ref{sec:landslide-directly}} aficionado).
		It exports several key environment variables used by the other scripts,
		ensures the instrumentation is up-to-date,
		and launches the simulator.
	\item {\tt getfunc.sh} defines several functions commonly used by {\tt build.sh} and {\tt definegen.sh} to extract function or global variable addresses from the program binary.
	\item {\tt symbols.sh} defines the names of kernel functions that can be instrumented automatically without a corresponding manual annotation (e.g., {\tt malloc} and friends, the names of the {\tt tell\_landslide} family, various library helpers such as {\tt panic}).
	\item {\tt build.sh} ensures the build of Landslide is up-to-date,
		and processes any dynamic configuration options which don't require updating the build (\sect{\ref{sec:landslide-dynamicconfig}})
		It verifies all required {\tt tell\_landslide} annotations are present,
		verifies all required {\tt config.\allowbreak{}landslide} options,
		processes the dynamic config options,
		checks whether or not {\tt definegen.sh} needs to be run again (via hashes stored in {\tt student\_specifics.h} of the program binary and static config options),
		and does so if necessary.
	\item {\tt definegen.sh} produces the
		%automatically-generated
		content of {\tt student\_specifics.h}.
		It repeatedly invokes the helpers defined in {\tt getfunc.sh}
		to find the addresses of both functions specified in the config options
		and functions whose names are known in advance.%
		\footnote{You might think it should invoke objdump but once and keep the output in a shell variable,
		but I tried that and it was mysteriously slower, so I gave up without ever figuring out why.}
\end{itemize}

The final output of these scripts is an auto-generated header, {\tt student\_specifics.h},
containing a bunch of {\tt \#define}s of the addresses of important functions in the compiled binary,
specific features enabled or disabled by the static config options (\sect{\ref{sec:landslide-staticconfig}}),
and so on.
The files {\tt kernel\_specifics.c}, {\tt user\_specifics.c}, and {\tt student.c} provide several interface functions
for interpreting the current program state with respect to these values.

\section{Algorithms of note}

\subsection{Kernel annotations}
\subsection{Bug identification}
\subsection{Data race analysis}
\subsection{Preemption point identification}
\label{sec:landslide-pps}
% TODO: talk about how w/in func and w/out func are processed
\subsection{State space estimation}
\subsection{Dynamic Partial Order Reduction}
\label{sec:landslide-dpor}
\subsection{Iterative Context Bounding}
\label{sec:landslide-icb}
\subsection{Heuristic loop, deadlock, and synchronization detection}
