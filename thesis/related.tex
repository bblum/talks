\chapter{Related Work}
\label{chap:relatedwork}
%\inspirationalquote{
%\begin{tabular}{p{0.58\textwidth}}
%To test if your paper makes a genuine contribution to its discipline,
%see if you can afford a generous tone in the "Related Work" section.
%\end{tabular}}
%{Conor McBride}

\inspirationalquote{
\begin{tabular}{p{0.51\textwidth}}
It is important to draw wisdom from many different places.
If you take it from only one place, it becomes rigid and stale.
\end{tabular}
}
{Iroh, Avatar: The Last Airbender}

Had I a dollar for every programmer before me who thought to ``solve'' concurrency with a perfect debugging tool,
well,
it probably would not be quite enough to retire on.
Even so,
this field is built of the contributions of many a brilliant mind
trying to carve out a presentable space in an overall impossible problem,
each making their own tradeoffs along the way.
While the previous chapters cited prior work as necessary in background discussions, algorithm descriptions, and so on,
this chapter aims to comprehensively tour the field,
orienting the reader's understanding of Landslide in the space of said tradeoffs.

\section{Systematic Concurrency Testing}

Equal partners in concurrency testing are the practical and the theoretical:
the former meaning tool implementations that target specific problem domains and help users as best one can,
and the latter meaning algorithmic advances to bring ever-larger state spaces within the realm of computational feasibility.
I discuss my most closely related works split in two sections accordingly.

\subsubsection{Tools}

Systematic concurrency testing dates back to Verisoft \cite{verisoft},
the 1997 tool which first attempted to exhaustively explore the possibile ways to interleave threads.
Since then, researchers have built many tools along the same lines to test many kinds of programs.
One of the best-known SCTs is Microsoft Research's CHESS \cite{chess},
a checker for userspace C++ programs which preempts on synchronization APIs by default,
supporting compiler instrumentation to preempt on memory accesses as well,
and which pioneered the ICB search strategy discussed below.

Many checkers exist which target programs written for various different types of concurrent environments.
MaceMC \cite{macemc}, MoDist \cite{modist}, SAMC \cite{samc}, ETA \cite{dbug-retreat}, and Concuerror \cite{concuerror},
focus on distributed systems, where concurrent events are limited to message-passing and may span across multiple machines.
R4 \cite{r4} and EventRacer \cite{eventracer} check event-driven concurrent programs typical in mobile applications.
dBug \cite{dbug-ssv}, another CMU original similar to CHESS,
integrates with Parrot \cite{parrot}, a determinizing runtime scheduler,
to approach state space reduction from a different angle by limiting which thread interleavings are possible to begin with.

Other checkers target specific programming languages' concurrency models and/or thread communication APIs.
Relacy \cite{relacy} is a header-only C++ SCT library for verifying synchronization primitives.
It instruments the existing C++ atomics API,
although requires custom annotations for other types of memory accesses,
and notably includes relaxed memory nondeterminism in its concurrency model.
Inspect \cite{inspect} uses a static alias analysis to instrument and preempt all memory accesses to potentially-shared data.
SPIN \cite{spin} defines the PROMELA domain-specific language
to specialize in verifying synchronization primitives such as RCU \cite{rcu}.
%
D\'{e}j\`{a} Fu \cite{dejafu} is a SCT
for the Haskell language,
whose strong type system guarantees that thread communication be confined to trusted, type-safe APIs.
%to a trusted API that implements internal synchronization to preserve type safety.
%Supporting both abstraction reduction and STM,
D\'{e}j\`{a} Fu instruments these interfaces (STM among them)
to check for deadlocks or nondeterministic behaviour in general,
which may arise despite the static guarantee of no data races.
The Rust language \cite{rust-language}
provides a more C++-like type system that also
%, which I contributed to personally,
achieves the same guarantee, although I know of no existing SCT for it yet.

If I might indulge by listing Landslide in its own related work section \cite{this-thesis},
I would distinguish it by its ability to find data-race preemption points via dynamic memory tracing,
rather than relying on user annotations or imprecise compiler instrumentation
as other tools do.
Compared to all other tools I know of,
it implements a wider range of state space explosion coping techniques,
some theoretical and some heuristic,
to help the user receive meaningful results as promptly as possible.
Its choice of a familiar pthread-like C threading API makes it suitable for inexpert users,
and its recent extension to HTM shows that it can support more modern concurrency patterns as well.

\subsubsection{Algorithms}

% transdpor - older than optimal dpor but still part of history?
% http://mir.cs.illinois.edu/marinov/publications/TasharofiETAL12TransDPOR.pdf


% effective stateless model checking for c/c++ concurrency - popl 18

% Surveying concurrency bug detectors based on types of detected bugs - idk what this is, from an email


%%%% non-smc testing tools -- even worth talking abt?

% jepsen-io/jepsen

\section{Data race analysis}

\section{Educational debugging}

Willgrind \cite{willgrind}

\section{Transactional memory}

% cchtm - the intel ismm17 paper on nvram htm
% stamp (mb just cite from txn section)

% transactions in relaxed mem architecutres - popl 18

% pldi 2018
% The Semantics of Transactions and Weak Memory in x86, Power, ARM, and C++
