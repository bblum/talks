\chapter{Related Work}
\label{chap:relatedwork}
%\inspirationalquote{
%\begin{tabular}{p{0.58\textwidth}}
%To test if your paper makes a genuine contribution to its discipline,
%see if you can afford a generous tone in the "Related Work" section.
%\end{tabular}}
%{Conor McBride}

\inspirationalquote{
\begin{tabular}{p{0.51\textwidth}}
It is important to draw wisdom from many different places.
If you take it from only one place, it becomes rigid and stale.
\end{tabular}
}
{Iroh, Avatar: The Last Airbender}

Had I a dollar for every programmer before me who thought to ``solve'' concurrency with a perfect debugging tool,
well,
it probably would not be quite enough to retire on.
At any rate,
this field is built of the contributions of many a brilliant mind
trying to carve out a presentable space in an overall impossible problem,
each making their own tradeoffs along the way.
While the previous chapters cited prior work as necessary in background discussions, algorithm descriptions, and so on,
this chapter aims to comprehensively tour the field,
orienting the reader's understanding of Landslide in the space of said tradeoffs.

\section{Systematic Concurrency Testing}

Equal partners in concurrency testing are the practical and the theoretical:
the former meaning tool implementations that target specific problem domains and help users as best one can,
and the latter meaning algorithmic advances to bring ever-larger state spaces within the realm of computational feasibility.
I discuss my most closely related works split in two sections accordingly.

\subsubsection{Tools}

Systematic concurrency testing dates back to Verisoft \cite{verisoft},
the 1997 tool which first attempted to exhaustively explore the possibile ways to interleave threads.
Since then, researchers have built many tools along the same lines to test many kinds of programs.
One of the best-known SCTs is Microsoft Research's CHESS \cite{chess},
a checker for userspace C++ programs which preempts on synchronization APIs by default,
supporting compiler instrumentation to preempt on memory accesses as well,
and which pioneered the ICB search strategy discussed below.

Many checkers exist which target programs written for various different types of concurrent environments.
MaceMC \cite{macemc}, MoDist \cite{modist}, SAMC \cite{samc}, ETA \cite{dbug-retreat}, and Concuerror \cite{concuerror},
focus on distributed systems, where concurrent events are limited to message-passing and may span across multiple machines.
R4 \cite{r4} and EventRacer \cite{eventracer} check event-driven concurrent programs typical in mobile applications.
dBug \cite{dbug-ssv}, another CMU original similar to CHESS,
integrates with Parrot \cite{parrot}, a determinizing runtime scheduler,
to approach state space reduction from a different angle by limiting which thread interleavings are possible to begin with.

Other checkers target specific programming languages' concurrency models and/or thread communication APIs.
Inspect \cite{inspect} uses a static alias analysis to instrument and preempt all memory accesses to potentially-shared data.
SPIN \cite{spin} tests algorithms defined in the PROMELA domain-specific language,
instruments every memory access, and
specializes in verifying synchronization primitives such as RCU \cite{rcu}.
%
D\'{e}j\`{a} Fu \cite{dejafu} is a SCT
for the Haskell language,
whose strong type system guarantees that thread communication be confined to trusted, type-safe APIs.
%to a trusted API that implements internal synchronization to preserve type safety.
%Supporting both abstraction reduction and STM,
It instruments these interfaces (STM among them)
to check for deadlocks or nondeterministic behaviour in general,
which either may arise despite the static no-data-race guarantee.
The Rust language \cite{rust-language}
provides a more C++-like type system that also
%, which I contributed to personally,
achieves the same guarantee, although I know of no existing SCT for it yet.

The problem of relaxed memory nondeterminism alone has inspired the creation of several new SCTs in the past few years.
Relacy \cite{relacy}, a header-only C++ SCT library for checking synchronization primitives,
was the first to broach this field,
although requires custom annotations for non-atomic memory accesses
and
%(according to later citing papers) % cdschecker
does not fully model all possible relaxed memory behaviours.
%designed for verifying synchronization primitives,
CDSChecker \cite{cdschecker} extends DPOR with a {\em reads-from} relation
to capture most of the C++11 memory model's new behaviours.
Nidhugg \cite{nidhugg} is a SCT for TSO and PSO which instruments LLVM abstract assembly,
% i don't actually know what the difference is #overlyhonestmethods
although does not yet support the C++11 memory model.
%% actually it doesn't -- they say they take the Source DPOR worse version. im confus?
% and extends the Optimal DPOR algorithm (discussed below) to include store buffer nondeterminism.
rInspect \cite{tsopso}
%models TSO and PSO differently, using shadow threads, and
offers further heuristic state space reduction using buffer bounding (described below).
RCMC \cite{rcmc} models a ``repaired'' version of the C++11 memory model known as RC11 \cite{rc11},
and professes to achieve the best state space reduction to date.
These tools each use various heuristics to account for spin-wait loops,
ranging from delay bounding \cite{bpor} to a rigid rewrite rule,
and provide only limited support so far for read-modify-write atomics
(at best, supporting them by introducing % oops
some redundant exploration).
%
No relaxed-memory SCT has yet proposed a satisfactory model for the ``thin-air'' problem \cite{sully-thesis},
which can cause state space cycles in a way not yet well-understood and remains future work.
%and notably includes relaxed memory nondeterminism in its concurrency model.
They also identify all data races
% TODO check if ther'es not a better section reference
(by the C++ definition rather than \sect{\ref{sec:quicksand-soundness}}'s)
as bugs immediately, rather than checking them for benign or buggy outcomes.
All of these tools are notably open-source -- an encouraging recent trend in the field.


If I might indulge by listing Landslide in its own related work section \cite{this-thesis},
I would distinguish it by its ability to find shared memory preemption points via dynamic tracing,
rather than relying on user annotations or imprecise compiler instrumentation
as other tools do.
Compared to all other tools I know of,
it implements a wider range of exponential explosion coping techniques,
some theoretical and some heuristic,
some inherited and some novel,
to help the user receive meaningful results as promptly as possible.
Its choice of a familiar pthread-like synchronization API makes it suitable for inexpert users,
and its recent extension to HTM adds support for more modern concurrency patterns as well.

% TODO
% Surveying concurrency bug detectors based on types of detected bugs - idk what this is, from an email

\subsubsection{Algorithms}

To date a number of techniques have been proposed to mitigate exponential explosion,
the Sisyphean rock of SCT.
The notion that some interleavings of concurrent threads could lead to indistinguishable program states and be therefore redundant,
known as {\em partial order reduction} (POR),
was first proposed in \cite{partial-model-checking}
and explored in detail in \cite{partial-order-methods}.
{\em Dynamic POR} (DPOR) was later developed in \cite{dpor},
proposing to track communication events between threads on-the-fly (i.e., dynamically)
rather than to rely on imprecise static alias analyses,
and is now considered the baseline for all subsequent state space reduction approaches in SCT.
That paper includes the {\em sleep sets} extension,
which Landslide includes in its implementation.
It is a {\em sound} reduction algorithm, meaning it will never fail to test a possible program behavior, despite skipping many execution sequences.
\sect{\ref{sec:landslide-dpor}} provides a detailed walk-through of how DPOR works,
as many of this thesis's contributions build directly upon it.

DPOR has since been extended in several ways to achieve further reduction
and to incorporate new concurrency models.
Optimal DPOR \cite{optimal-dpor} extends sleep sets into the more expressive {\em wakeup trees},
which provably tests exactly one interleaving from each equivalence class,
i.e., the optimal possible reduction,
at least under the memory independence definition of equivalence.
Extending the equivalence relation itself to capture not just memory {\em address} conflicts
but also the {\em values} read and written,
SATCheck \cite{satcheck} and Maximal Causality Reduction (MCR) \cite{mcr}
use an SMT solver \cite{z3} to identify additional pruning opportunities.
Implementing wakeup trees or SMT-driven exploration in Landslide is left to future work.

Several other recent advances extend DPOR to new concurrency models,
beyond the shared-memory-threading model outlined in \sect{\ref{sec:landslide-dpor}}.
TransDPOR \cite{transdpor} provides extra domain-specific reduction for message-passing actor programs
by exploiting the fact that the dependency relation is transitive in the absence of shared state.
The $R^4$ algorithm \cite{r4} (corresponding to the R4 checker mentioned above)
extends DPOR to event-driven programs by separating the notion of enabled events from that of multiple threads.
DPOR for TSO and PSO \cite{tsopso}
extends the concurrency model
using {\em shadow threads}, which interleave with traditional threads to represent store buffer nondeterminism,
which can expose bugs not even possible in the strong consistency model
such as discussed in \sect{\ref{sec:tm-warpzone-relaxed}}.
It also introduced a heuristic {\em buffer bounding} technique, analogous to ICB,
to mitigate the corresponding increase in state space size.
The same year, Nidhugg \cite{nidhugg} proposed a DPOR extension to account for TSO and PSO
using {\em chronological traces}.
MCR was recently extended to support relaxed memory models likewise \cite{mcr-tsopso}.
Just this year, RCMC \cite{rcmc} proposed to replace the interleaving model entirely with {\em execution graphs},
offering further reduction still.
Somewhat analogously for HTM, this work's Chapter~\ref{chap:tm}
extended DPOR's concurrency model to include failure injection,
% not "new" -- abstraction reduction is the same as always
and proposed three reduction strategies, one sound and two heuristic,
%to make some progress up the exponential mountain.
to keep state spaces manageable.

Of course, no matter how optimal a sound reduction, there will always be programs too large to test.
To provide even partial results for state spaces that exceed the testing budget
(whether as predicted by automatic estimation \cite{estimation} or by a human's wild guess),
various heuristic exploration strategies have been proposed.
Preemption Sealing \cite{sealing} allows programmers to manually exclude preemption points
arising from trusted source code modules;
Landslide implements this as the {\tt without\_function} command (\sect{\ref{sec:landslide-pps}}).
Iterative Context Bounding (ICB) \cite{chess-icb} (\sect{\ref{sec:landslide-icb}})
orders the search space by increasing number of preemptions in each branch,
which is more likely to expose bugs sooner should they exist;
BPOR \cite{bpor} extends DPOR to preserve soundness
thereunder. % what a word
Landslide implements ICB and BPOR for Chapter~\ref{chap:quicksand}'s control experiments,
although does not yet incorporate it into this thesis's own contributions.
Chapter~\ref{chap:quicksand}'s Quicksand algorithm is, effectively, another such heuristic search strategy,
focusing on preemption point subsets rather than context switch bounding.
Each of these approaches is compatible (and indeed, throughout this thesis used often in concert)
with the sound reduction analyses listed above.
% mentioning PCT would disrupt the flow here and it's a dumb technique anyway so i won't bother

%%%% non-smc testing tools -- even worth talking abt?

% jepsen-io/jepsen

\section{Data race analysis}

\section{Educational debugging}

Willgrind \cite{willgrind}

\section{Transactional memory}

% cchtm - the intel ismm17 paper on nvram htm
% stamp (mb just cite from txn section)

% transactions in relaxed mem architecutres - popl 18

% pldi 2018
% The Semantics of Transactions and Weak Memory in x86, Power, ARM, and C++
