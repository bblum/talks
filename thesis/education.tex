\chapter{Education}
\label{chap:education}

\inspirationalquote{Knowing the students might one day
%find a way to
fix their concurrency bugs...
it fills you with determination.}{Undertale (paraphrased)}

Concurrency is taught in as many different ways as there are
systems programming classes at universities which teach the subject.
Yet one thing they all have in common is presenting the concurrency bug
as some elusive menace,
humanity's best weapon against which is mere random stress testing.
This chapter will prove stateless model checking's mettle as a better alternative in the educational theatre.

While the previous chapter demonstrated Landslide's bug-finding power
compared to prior MC techniques in a controlled environment,.
whether it offers pedagogical merit in the hands of students and/or TAs is a separate question.
And while my MS thesis \cite{landslide} showed that students
could annotate P3 Pebbles kernels and thence use Landslide to debug them,
the annotations alone required 2 hours of effort on average per user,
meaning the only students who could benefit were the ones already succeeding enough to have such free time.
% TODO: sect ref
Since then, I have extended Landslide with a fully-automatic instrumentation process
for Pebbles thread libraries (P2s) and Pintos kernels
to improve its accessibility.

I have run several user studies in the Operating Systems classes
at Carnegie Mellon University (CMU), University of Chicago (U. Chicago), and Penn State University (PSU),
wherein students get to use Landslide to find and diagnose their own bugs during the semester.
% TODO: put section refs
At CMU, I recorded logs and code snapshots as students used Landslide during P2.
At CMU and PSU, I surveyed students on their experience after submitting their Landslide-debugged P2s.
At U. Chicago, I collaborated with a TA to check submitted Pintos kernels,
then return any resulting bug reports to students and likewise survey them on the quality of the diagnostic output.

%%%%%%%%%%%%%%%%%%%%%%%%%%%%%%%%%%%%%%%%%%%%%%%%%%%%%%%%%%%%%%%%%%%%%%%%%%%%%%%%

\section{Pebbles}

% Possible experiment questions
% compare e.g. use after free bug reporce from OOPSLA data set, to P2 grade files, see who has thread exit uafs
% is landslide better at finding thread uafs than TAs
% same Q for other stuff.. (expect paraguay answer to be "no", explain why)

\subsection{Automatic instrumentation}

\subsection{Evaluation}

%%%%%%%%%%%%%%%%%%%%%%%%%%%%%%%%%%%%%%%%%%%%%%%%%%%%%%%%%%%%%%%%%%%%%%%%%%%%%%%%

\section{Pintos}

\subsection{Automatic instrumentation}

\subsection{Evaluation}

% survey questions i WISH i had asked
% - did you have any technical difficulties w landslide that i had to intervene on

% \section{Survey results} % ???
