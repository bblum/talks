\documentclass[xcolor=dvipsnames]{beamer}
\usepackage{lmodern}
\usepackage[T1]{fontenc}
\usepackage[english]{babel}
\usepackage[utf8]{inputenc}

\usepackage{manfnt}
\usepackage{wasysym}
\usepackage{listings}
\usepackage{graphicx}
\usepackage{url}
\usepackage{ulem}
\usepackage{marvosym}
\usepackage{proof}
\usepackage{array}
\setbeamertemplate{navigation symbols}{}

\title[Demystifying Debugging]{{\bf Demystifying Debugging}}
\subtitle[]{{\em when getting it right the first time every time fails}}
\author[Ben Blum]{Ben Blum \texttt{(bblum@andrew.cmu.edu)}}

\institute[98-172]{Great Practical Ideas for Computer Scientists}
\date[]{2012, November 1}

\setbeamertemplate{footline}{\hspace*{.5cm}\scriptsize{\insertauthor\hspace*{50pt} \hfill\insertframenumber\hspace*{.5cm}}} 

\usecolortheme{seahorse}
\usecolortheme{rose}
\useoutertheme{infolines}

\usecolortheme[named=RoyalBlue]{structure}

\newcommand\noob{\mathsf{noob}}
\newcommand\gibs{\mathsf{gibs}}
\newcommand\dps{\mathsf{dps}}
\newcommand\squig\rightsquigarrow
\newcommand\Coloneqq{\mathrel{\mathop{::}}=}
\newcommand\dmg{\text{\Laserbeam}}
\newcommand\delter\delta
\newcommand\alpher\alpha
\newcommand\defnor{\text{ }|\text{ }}

\newcommand\pimp{\mathop{\supset}}
\newcommand\pand{\mathop{\wedge}}
\newcommand\por{\mathop{\vee}}
\newcommand\ptrue{\top}
\newcommand\pfalse{\bot}


\begin{document}
% \renewcommand{\inserttotalframenumber}{28} % If you want to hide bonus slides
\normalem
\begin{frame}
	\titlepage
\end{frame}

%%%%%%%%%%%%%%%%%%%%%%%%%%%%%%%%%%%%%%%%%%%%%%%%%%%%%%%%%%%%%%%%%%%%%%%%%%%%%%%%
%%%%%%%%%%%%%%%%%%%%%%%%%%%%%%%%%%%%%%%%%%%%%%%%%%%%%%%%%%%%%%%%%%%%%%%%%%%%%%%%
%%%%%%%%%%%%%%%%%%%%%%%%%%%%%%%%%%%%%%%%%%%%%%%%%%%%%%%%%%%%%%%%%%%%%%%%%%%%%%%%

\newcommand\linegap{\vspace{0.2in}}
\newcommand\breakslide[1]{\begin{frame}{} \begin{center} \Large #1 \end{center} \end{frame}}
\newcommand\related[1]{\textsuperscript{\em [#1]}}
\newcommand\hilight[2]{\color{#1}#2\color{black}}

\begin{frame}{Outline}
	\textbf{Why teach debugging?}
		% brief motivation
		% toolbox
	\linegap

	\textbf{``Tell Me a Story''}
	\begin{itemize}
		\item The good debugger's attitude
			% "get more operators"
		\item Binary Search
	\end{itemize}
	\linegap

	{\bf Two Main Approaches}
	\begin{itemize}
		\item Time Travel
			% small test cases / minimize the test case
		\item Space Travel
			% printf debugging
			% invariant checkers and asserts
	\end{itemize}
	\linegap

	{\bf Assorted Tricks}
	\begin{itemize}
		\item Question assumptions
		\item Collect data
	\end{itemize}
\end{frame}

%%%%%%%%%%%%%%%%%%%%%%%%%%%%%%%%%%%%%%%%%%%%%%%%%%%%%%%%%%%%%%%%%%%%%%%%%%%%%%%%
\section{Motivation}
%%%%%%%%%%%%%%%%%%%%%%%%%%%%%%%%%%%%%%%%%%%%%%%%%%%%%%%%%%%%%%%%%%%%%%%%%%%%%%%%

\begin{frame}{``The Universal Backup Plan''}
	\textbf{15-112, 15-122, 15-150, ...}
	\begin{itemize}
		\item What do these classes all have in common?
		\pause
		\item How to write code that {\em works}
	\end{itemize}
	\pause
	{\bf 98-172}
	\begin{itemize}
		\item What to do with code that {\em doesn't}
	\end{itemize}
	% TODO: a picture goes here -- blank slate vs equipped/prepared technician
\end{frame}

\begin{frame}{Take-away}
	\textbf{Debugging toolbox}
	\begin{itemize}
		\item Two major {\em strategies} for approaching bugs
		\item Assorted {\em tactics} for applying to broken programs
	\end{itemize}
	% Say: "It's not, e.g., a checklist of bugs that you can just run down and test one-by-one; it's more abstract than that. You have to use your brain to figure out how to apply each tool. But the hope is that you won't feel lost staring at a blank slate of possible approaches.
\end{frame}


%%%%%%%%%%%%%%%%%%%%%%%%%%%%%%%%%%%%%%%%%%%%%%%%%%%%%%%%%%%%%%%%%%%%%%%%%%%%%%%%
\section{Storytelling}
%%%%%%%%%%%%%%%%%%%%%%%%%%%%%%%%%%%%%%%%%%%%%%%%%%%%%%%%%%%%%%%%%%%%%%%%%%%%%%%%

\begin{frame}{The Essence of Debugging}
	...is {\bf telling a good story.}
	\pause
	\linegap
	Two stories, actually:
	\begin{itemize}
		\item What you wanted to happen
		\item What the computer actually did
	\end{itemize}
	\pause
	Your bug: The point where the stories diverge
	\begin{itemize}
		\item Can you describe that point in as much detail as possible?
	\end{itemize}
\end{frame}

\begin{frame}{The Essence of Debugging}
	Because you will write unique bugs, you will be telling a story nobody else knows.

	\linegap
	``Plot summaries'' are often not enough: My program crashes...
	\pause
	\begin{itemize}
		\item with a segmentation fault...
		\pause
		\begin{itemize}
			\item in \texttt{take\_final\_exam()}...
			\pause
			\begin{itemize}
				\item because \texttt{study\_for\_final()} passed \texttt{NULL} to it.
				\item There we go!
			\end{itemize}
		\end{itemize}
	\end{itemize}
\end{frame}

\begin{frame}{Asking for Help}
	This DOESN'T mean...
	\begin{itemize}
		\item That TAs can't help you without a detailed story
		\item (That you shouldn't seek help until you've already found your bug?!)
	\end{itemize}
	\pause
	This does mean...
	\begin{itemize}
		\item You need a detailed story to understand your bug.
		\item If the TA doesn't know the story, how can they help?
		\pause
		\begin{itemize}
			\item They know what you should do to make your story better.
		\end{itemize}
	\end{itemize}
\end{frame}

\begin{frame}{Running Example}
	% TODO
\end{frame}

%%%%%%%%%%%%%%%%%%%%%%%%%%%%%%%%%%%%%%%%%%%%%%%%%%%%%%%%%%%%%%%%%%%%%%%%%%%%%%%%
\section{Main Approach}
%%%%%%%%%%%%%%%%%%%%%%%%%%%%%%%%%%%%%%%%%%%%%%%%%%%%%%%%%%%%%%%%%%%%%%%%%%%%%%%%

\subsection{Time Travel}

\breakslide{{\bf When} things go wrong\dots}

\begin{frame}{Telling a Story of Time}
	Remember, two stories: Expected outcome vs Actual outcome
	\linegap
	What does this mean 
	% Text: "First I ... then I ... but then it ...!"
\end{frame}

\begin{frame}{Print debugging}
	% Insert some code: the test case with print statements inserted in it
	% Insert a picture: the time-cloud with some printed-out things in it
	Should usually be the first thing you try!
\end{frame}

\begin{frame}{Asserts: Stealthier, more concealable prints}
\end{frame}

\subsection{Space Travel}

\breakslide{{\bf Where} things go wrong\dots}

\begin{frame}{Telling a Story of Space}
	Remember, two stories: Expected outcome vs Actual outcome
	\linegap
\end{frame}

\end{document}
%%%%%%%%%%%%%%%%%%%%%%%%%%%%%%%%%%%%%%%%%%%%%%%%%%%%%%%%%%%%%%%%%%%%%%%%%%%%%%%%
% vim: foldmethod=indent
