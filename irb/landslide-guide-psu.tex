\documentclass{article}
%\usepackage{amsmath,amsthm,amssymb,fullpage,yfonts,graphicx,proof,subfig,wrapfig,appendix,hyperref,mdwlist,wasysym}
\usepackage{amsmath,amsthm,amssymb,fullpage,yfonts,graphicx,proof,appendix,hyperref,mdwlist,wasysym}
\usepackage{upgreek}
%\usepackage{times}
\usepackage[charter]{mathdesign}
\usepackage{hyperref}
\usepackage{algorithm}
\usepackage{algpseudocode}
\usepackage{multirow}
\usepackage[usenames,dvipsnames]{xcolor}
%\usepackage{epsfig}
\usepackage[bottom]{footmisc}
%\usepackage{mjz-titlepage}
\usepackage{framed}
\usepackage{setspace}
%\setstretch{1.05}
\usepackage{subfig}
\usepackage{changebar}
\usepackage{colortbl}
\usepackage{wrapfig}
\usepackage{xspace}

\newcommand\true{\;\textit{true}}
\newcommand\false{\;\textit{false}}

\newcommand\alpher\alpha
\newcommand\beter\beta
\newcommand\gammer\gamma
\newcommand\delter\delta
\newcommand\zeter\zeta
\newcommand\Sigmer\Sigma

\newcommand\NN{\mathbb{N}}
\newcommand\QQ{\mathbb{Q}}
\newcommand\RR{\mathbb{R}}
\newcommand\ZZ{\mathbb{Z}}

\begin{document}
%\captionsetup{width=.75\textwidth,font=small,labelfont=bf}

\newcommand\classname{CMPSC 473\xspace}
\title{\bf Landslide: Systematic Testing in \classname \\ User Guide}
\author{Ben Blum}
\date{}
\maketitle

\section{Introduction}

You should have received a document titled {\em User Study Information Sheet} that describes your rights as a study
participant and our confidentiality procedures. If you are missing this, please contact me immediately. My email
address is {\tt bblum@cs.cmu.edu}.
\\

\noindent Likewise, please make sure you have read and understand the lecture slides on systematic testing, which are
available on the \classname course website. If you have any questions about the research background I would be
happy to answer them.

\section{Instructions}

We encourage you to use Landslide on the Vagrant VM or the W204 cluster machines.
Landslide has not been tested with other configurations and may fail
mysteriously on other development environments, and we cannot promise to be of any help.

\begin{enumerate}
	\item Clone the repository - {\tt git clone https://github.com/bblum/landslide.git} - this will be your workspace.
		On the Vagrant VM, do this in the home directory, {\tt \textasciitilde{}/},
		{\em not} in the shared host directory {\tt /vagrant}.
	\item Run {\tt ./psu-setup.sh /vagrant},
		or if using on the W204 cluster, replace {\tt /vagrant} with the absolute path to your project directory.
		\begin{itemize}
			\item This command will import your project into the {\tt ./pebsim/p2-basecode/} subdirectory of the landslide repository and then attempt to build it.
				Whenever you update your thread library code, please do so in its original {\tt /vagrant} location
				and then run {\tt psu-setup.sh} again.
			%\item {\tt /vagrant} should be the path with {\tt 410kern}, {\tt 410user}, {\tt user/libthread}, etc as subdirectories, not the {\tt user/libthread} directory itself.
		\end{itemize}
	\item Run {\tt ./landslide OPTIONS} to run tests with Landslide, replacing {\tt OPTIONS} with any of the following:
		\begin{enumerate}
			\item Use {\tt ./landslide -h} to view command line options. The most important ones are {\tt -t}, max testing
				time, and {\tt -p}, which test program to run.
			\item Suggested test configurations to run are as follows.
				You can, of course, change the time limits or test programs as you see fit.
				%For the longer-running configurations,
				%we recommend running them inside of a {\tt screen} or {\tt tmux} session.
				\begin{enumerate}
					\item Atomic tests
					\begin{itemize}
						\item \texttt{./landslide -t 30m -p atomic\_fetch\_add}
						\item \texttt{./landslide -t 30m -p atomic\_fetch\_sub}
						\item \texttt{./landslide -t 30m -p atomic\_exchange}
						\item \texttt{./landslide -t 30m -p atomic\_compare\_swap}
					\end{itemize}
					\item Threading tests
					\begin{itemize}
						\item \texttt{./landslide -t 30m -p thr\_exit\_join}
						\item \texttt{./landslide -t 30m -p mutex\_test}
						\item \texttt{./landslide -t 30m -p paradise\_lost}
						\item \texttt{./landslide -t 30m -p broadcast\_test}
						\item \texttt{./landslide -t 30m -p paraguay}
						\item \texttt{./landslide -t 30m -p rwlock\_downgrade\_read\_test}
						%	\\
						%\item \texttt{./landslide -t 3h -p thr\_exit\_join}
						%\item \texttt{./landslide -t 3h -p mutex\_test}
						%\item \texttt{./landslide -t 3h -p paradise\_lost}
						%\item \texttt{./landslide -t 3h -p broadcast\_test}
						%\item \texttt{./landslide -t 3h -p paraguay}
						%\item \texttt{./landslide -t 3h -p rwlock\_downgrade\_read\_test}
							\\
						\item \texttt{./landslide -t 8h -p thr\_exit\_join}
						\item \texttt{./landslide -t 8h -p mutex\_test}
						\item \texttt{./landslide -t 8h -p paradise\_lost}
						\item \texttt{./landslide -t 8h -p broadcast\_test}
						\item \texttt{./landslide -t 8h -p paraguay}
						\item \texttt{./landslide -t 8h -p rwlock\_downgrade\_read\_test}
					\end{itemize}
				%\item To build your own custom test case, %for use with Landslide,
				%	put it in the {\tt pebsim/p2-basecode/user/progs} subdirectory %of the repository
				%	and edit the {\tt STUDENTTESTS} line in {\tt pebsim/p2-basecode/config-incomplete.mk}
				%	(note: NOT {\tt config.mk}; the setup script auto-generates that file).
				%	Then run {\tt ./psu-setup.sh} again.
				%	Refer to slide 37 of the Landslide lecture to make sure your test is ``Landslide-friendly''.
				\end{enumerate}
			\item Landslide's bug reports will show up in the current directory as HTML files which you can view in a web browser.
				\begin{itemize}
					\item On the Vagrant VM, move these files to {\tt /vagrant}, the shared directory, and open them with a web browser on the host.
					\item On the W204 cluster, you should be able to open the files directly from the landslide repository.
				\end{itemize}
				Landslide may also print some {\em data races},
				even if it doesn't emit an HTML bug report,
				for example as follows:
				\[
					\texttt{Found a racy access at 0x0100291e in atomic\_compare\_swap (atomic.S:42)}
				\]
				These are not necessarily bugs; please refer to slide 29 of the Landslide lecture for details about what this means.
		\end{enumerate}
	\item After fixing any found bugs, re-run Landslide with the same test options to confirm that your fix works.
		(Remember to re-run {\tt psu-setup.sh} as in step 2 to make sure Landslide sees your updates.)
\end{enumerate}

\noindent {\bf Note on simultaneous Landslides:} Landslide does not support running multiple tests at once;
it needs to create some temporary files to annotate your code, and multiple instances of Landslide can step on each other's toes by clobbering those files.
%If you want to try multiple tests at once, simply {\tt git clone} the repository again in a different directory, and run one instance of Landslide in each.
\\

% TODO
% \nointend{\bf Note on disk usage:}

\noindent {\bf Asking questions:} If you have any techical questions about Landslide,
such as how to interpret its html interleaving traces,
how to write a custom test case, etc.,
feel free to email me ({\tt bblum@cs.cmu.edu}).
I will {\em not} provide advice on the thread library project itself,
nor will I personally look at your code to help you fix your bugs.
%Once Landslide shows you a specific thread interleaving that led to a bug, the rest is in your hands.
%However, we cannot provide help on how to understand why your bug happened or how to fix it.
The point of
the study is that Landslide's automation plus your own brain should be enough!
On the other hand, if you have
a design question while deciding between potential ways to fix a bug, send it to \classname course staff (not me) as usual.
\\

\noindent {\bf If something goes wrong:} If you find a Landslide bug, such as an assertion/crash (it will say ``Landslide crashed. This is not your fault.''), or such as a
bug report that you think is actually a bug in Landslide rather than a bug in your P2, or such as it getting stuck
and not exiting in the given time limit, please create a tarball of your workspace
using the command {\tt tar cjvf landslide-crash.tar.bz2 ./},
%, just as you would to report a reference kernel bug,
email it to me ({\tt bblum@cs.cmu.edu}), and I'll try to fix your issue promptly.
%For any other technical
%issues getting it to work, or if you have difficulty understanding a bug report, send email to .
\\

\end{document}
