\chapter{Pintos}
\label{chap:pintos}

The Pintos architecture \cite{pintos}, outlined in Section~\ref{sec:overview-pintos},
provides an opportunity to test Landslide's pedagogical mettle beyond CMU's walls.
Supporting Pintos as well as Pebbles will broaden Landslide's impact considerably,
%as it will prove that the previous two chapters' techniques apply generally...
as Pintos is already used by many more universities than Pebbles.

The scheduler and concurrency primitives which Pintos's basecode provides
limit the remaining concurrency-critical code left for students to write,
which limits the degree to which Landslide can test student submissions.
This is in contrast to the Pebbles assignments at CMU, in which students must implement all the functionality listed in Section~\ref{sec:overview-pintos}.
On the other hand, the upside of Pintos providing substantially more basecode than Pebbles
is that most, if not all, of Landslide's kernel instrumentation can be done automatically,
using the names of the core scheduler functions already provided.
For the experiments in Chapter~\ref{chap:quicksand}, I already implemented rudimentary support to test Pintos kernels, but more remains to be done.

\section{Existing Progress}

So far, just enough Pintos support exists to have run Chapter \ref{chap:quicksand}'s experiments. This includes:
%The existing Pintos support includes
\begin{itemize}
	\item a new set of annotations (e.g., {\tt mutex\_lock} in Pebbles is now called {\tt sema\_down};
	\item handling several quirks of the basecode's scheduling behaviour; %(ask me if you really care to know);
	\item extending the heap tracker to handle Pintos's page allocator {\tt palloc} as well as the kernel {\tt malloc},
		as well as the fact that kernel memory is not direct-mapped like in Pebbles;
	\item new thread-liveness code to detect when a test case terminates;
	\item skipping some busy-wait loops in the boot sequence used for device communication;
	\item an extra fall-back case in Landslide's timer injection algorithm; %the workings of which I don't even remember anymore,
	\item and extending the lock-set and vector-clock analyses to include disabling interrupts.
\end{itemize}
%(which was actually pretty tricky when using vector clocks for pure-HB).
All these features apply to behaviours included with the stock base-code, so should apply generally to all but the most adventurous student implementations.

\section{Proposed Work}

%However, as you might have guessed,
There are still a few cases where the existing instrumentation is not fully general.
When preparing Chapter~\ref{chap:quicksand}'s experiments, I manually adjusted some student submissions to be compatible with Landslide's existing annotation process.
In order to distribute Landslide for student use beyond CMU, it will need to support these cases automatically.
Examples of such ad-hoc fixes I would need to automate include:
\begin{itemize}
	\item Allowing for the priority scheduler's runqueues to be implemented as an array of queues, rather than the single queue head as provided in the basecode.
	\item Insert the {\tt tell\_landslide\_sleeping} annotation more intelligently than looking for a {\tt list\_insert\_ordered} call (a basecode function which most, but not all, students use).
	\item Some versions of the basecode distribute a {\tt sema\_up} implementation which {\tt yield}s unconditionally, which Landslide must bypass.
	\item Some students use {\tt timer\_sleep} or {\tt while (!flag) continue} when they should use {\tt yield}, which can lead to false-positive deadlocks if not automatically replaced.
	%\item I'm sure I've forgotten some less-common ones, which I expect to handle on-the-fly when students email me Landslide crash reports.
\end{itemize}
When fixing my sample of Pintoses by hand, there were also myriad deterministic bugs, such as use-after-frees e.g.~arising from unsafe {\tt strlen} calls.
As with Pebbles student experiments, however, I intend the students to encounter such bug reports and fix them on their own, despite not being concurrency bugs.
%It would be a Ph.D. unto itself to automatically ensure
Automatically ensuring stable determinized execution of arbitrary student code is beyond the scope of this thesis (if not outright impossible).
%Actually, probably outright impossible.

\subsection{New emulation platform}

Most practically, I will need to free Landslide from its dependence on Simics,
which requires paid licenses for its use. %beyond CMU's walls.
Other candidate emulation (simulation or virtualization) platforms for Landslide include Bochs, QEMU, VMWare, and Xen.
%I am presently studying which of these will most suit Landslide's needs,
%but so far Bochs seems like the best bet,
I have begun work porting Landslide to Bochs, which is open-source and provides both instruction and memory tracing, and
which the Berkeley, Stanford, and U. of Chicago classes already use as their simulator.

\subsection{Experimental goals}

%Having only two semesters to test Landslide with Pintos students,
%I'll be unable to gather as comprehensive a dataset as I have with 15-410 at CMU.
%This will serve more as a proof-of-concept, demonstrating that MC can realistically replace stress testing in operating systems courses worldwide.
%Nevertheless, I intend also to analyze what little data I can gather, comparing the incidences and types of bugs found between Pebbles and Pintos populations.
%This will perhaps shed light on the advantages and/or shortcomings of either project,
%leading to recommendations for improving either project's educational value (independently of using MC).

I aim to run a similar user study with Pathos students as I've done so far with Pebbles.
After finding one or more Pintos-using classes at other schools to collaborate with,
I will distribute Landslide to the students during the 'threads' and 'userprog' projects,
have them run a similar suite of tests to those in Chapter \ref{chap:quicksand}'s experiments,
and collect and analyze the results.
I plan to analyze the overall incidences and types of bugs found between Pebbles and Pintos populations,
which will perhaps shed light on the advantages and/or shortcomings of either project,
and lead to recommendations for improving either project's educational value (whether by incorporating MC or otherwise).
