\chapter{Background}
\label{chap:background}

This chapter will introduce the necessary background material on concurrency, stateless model checking, data-race analysis, and the relevant undergraduate operating systems classes.

\section{Concurrency Bugs}
% TODO

%%%%%%%%%%%%%%%%%%%%%%%%%%%%%%%%%%%%%%%%%%%%%%%%%%%%%%%%%%%%%%%%%%%%%%%%%%%%%%%%

\section{Stateless Model Checking}

Model checking \cite{verisoft} is a testing technique for systematically exploring the possible thread interleavings of a concurrent program.
A model checker executes the program repeatedly, each time according to a new thread interleaving, until the state space (or the CPU budget) is exhausted.
During each execution, it forces threads to execute serially, thereby confining the program's nondeterminism to controlled thread switches.
Some model checkers explicitly store the set of visited program states as a means of identifying equivalent interleavings \cite{spin}.
This approach is called {\em stateful} model checking.
In this thesis, I focus on {\em stateless} model checking,
which instead analyzes the sequence of execution events to avoid a prohibitive memory footprint.
Henceforth I will abbreviate ``stateless model checking'' simply as ``model checking'' for brevity.

{\bf Static versus dynamic analysis.}
Model checking is a {\em dynamic} program analysis, meaning that it observes the operations and accesses performed by the program as its code is executed.
In contrast, {\em static} program analyses check certain properties at the source code level.
Static analyses are ideal for ensuring certain standards of code quality, which often correlates with correctness,
but cannot decide for certain whether a given program will fail during execution without actually running the code.
% TODO: cite incompleteness theorem
Static analyses face the challenge of {\em false alarms} (or {\em false positives}):
code patterns which look suspicious but are actually correct.
A debugging tool which reports too many false alarms will dissuade developers from using it \cite{racerx}.
Dynamic analysis, our approach, identifies program behaviours that are definitely wrong,
so each bug report is accompanied by concrete evidence of the violation.
Assertions, segfaults, use-after-free of heap memory, and deadlock are examples of such failures we check for,
although a checker may also include arbitrary program-specific predicates.

{\bf Preemption points.}
During execution, a model checker identifies a subset of the program's operations as ``interesting'', i.e.,
points at which interrupting the current thread to run a different one is likely to produce different behaviour.
These points, called {\em preemption points}, may be identified by any combination of human intuition and machine analysis.
Typical preemption points include the boundaries of synchronization APIs (e.g., {\tt mutex\_lock}) or accesses to shared variables.
Considering that at each preemption point multiple threads exist as options to run next, the set of possible ways to execute the program can be viewed as a tree.
% TODO: fig tree
I show an example execution tree in Figure~\ref{fig:tree}.
The number of preemption points in each execution define the depth of this tree, and the number of threads available to run define the branching factor.
Hence, in a program with $n$ preemption points and $k$ threads available to run at each, the state space size is $n^k$.
Addressing the scaling problem in this exponential relation is the central research problem for all model checkers.

{\bf Reduction techniques.}
Dynamic Partial Order Reduction \cite{dpor} (henceforth, DPOR) is the premier algorithm for mitigating the exponential explosion that arises as program size increases.
It identifies equivalent execution sequences according to Mazurkiewicz trace theory \cite{mazurkiewicz},
and tests at least one execution from each equivalence class.
Intuitively, if two thread transitions between preemption points do not conflict on any shared resource access, reordering them produces an equivalent interleaving, i.e., the same program behaviour.
% TODO: put fig dpor
Figure~\ref{fig:dpor} shows a part of an execution tree in which the operations by threads 1 and 2 are independent, and hence only one of the subtrees need be tested.
Over the years, researchers have developed many enhancements to DPOR, such as Optimal DPOR \cite{optimal-dpor}, parallelizable DPOR \cite{parallel-dpor}, SAT-directed model checking \cite{satcheck}, Maximal Causality Reduction \cite{mcr}, and DPOR for relaxed memory architectures \cite{tsopso}.

{\bf Search heuristics.}
However, even though DPOR can prune an exponential number of redundant interleavings, the state space size is still exponential in the number of {\em dependent} (conflicting) interleavings.
Developers will always want to test larger and larger programs, so no matter the quality of our reduction algorithm,
we must accept that some tests will be too large to be fully tested in a reasonable time.
Hence, recent model checking research has turned to heuristic techniques for achieving further reduction,
optimizing the search to try to uncover bugs faster (should they exist)
at the expense of possibly missing other bugs,
or missing the chance to complete a full verification.
Iterative Context Bounding \cite{chess-icb} is a popular such technique which heuristically reorders the search to prioritize interleavings with fewer preemptions first.
This heuristic is based on the insight that most bugs require few preemptions to uncover, so interleavings with a number of preemptions that exceeds a certain bound will be de-prioritized, only tested until after all the fewer-preemption interleavings are completed.
Preemption sealing \cite{sealing} is another heuristic strategy which restricts the scope of the search by limiting the model checker to use only preemption points arising from certain functions in the source code.
This allows developers to vastly reduce state space size by identifying which program modules are already trusted,
although it requires some human intuition to correctly mark those boundaries.

%%%%%%%%%%%%%%%%%%%%%%%%%%%%%%%%%%%%%%%%%%%%%%%%%%%%%%%%%%%%%%%%%%%%%%%%%%%%%%%%

\section{Data Race Analsyis}

\begin{figure}[t]
        \small
	\begin{center}
\begin{tabular}{c}
\begin{tabular}{rll}
        & \multicolumn{2}{c}{\texttt{int x = 0; bool y = false; mutex\_t mx;}} \\
        \\
        & {\bf Thread 1} & {\bf Thread 2} \\
        1 & \texttt{\hilight{brickred}{x++;}~// A1} & \\
        2 & \texttt{mutex\_lock(\&mx);} & \\
        3 & \texttt{mutex\_unlock(\&mx);} & \\
        4 & & \texttt{mutex\_lock(\&mx);} \\
        5 & & \texttt{mutex\_unlock(\&mx);} \\
        6 & & \texttt{\hilight{brickred}{x++;}~// A2} \\
\end{tabular}
\\
	{\normalsize (a) True potential data race.}
\\
\\
\begin{tabular}{rll}
        %& \multicolumn{2}{c}{\texttt{int x = 0; bool y = false; mutex\_t mx;}} \\
        & {\bf Thread 1} & {\bf Thread 2} \\
        1 & \texttt{\hilight{brickred}{x++;}~// B1} & \\
        2 & \texttt{mutex\_lock(\&mx);} & \\
        3 & \texttt{y = true;} & \\
        4 & \texttt{mutex\_unlock(\&mx);} & \\
        5 & & \texttt{mutex\_lock(\&mx);} \\
        6 & & \texttt{bool tmp = y;} \\
        7 & & \texttt{mutex\_unlock(\&mx);} \\
        8 & & \texttt{if (tmp) \hilight{brickred}{x++;}~// B2} \\
        %8 & & \texttt{if (tmp)} \\
        %9 & & \texttt{~~~~\hilight{brickred}{x++;}~// B2} \\
\end{tabular}
\\
{\normalsize (b) No data race in any interleaving.}
\end{tabular}
	\end{center}
\caption{{Data-race analyses may be prone to either {\em false negatives} or {\em false positives}.
Applying Happens-Before to program (a) will miss the potential race possible between A1/A2 in an alternate interleaving,
while using Limited Happens-Before on (b) will produce a false alarm on B1/B2.}}
\label{fig:hb-example}
\end{figure}

Data race analysis \cite{eraser} identifies pairs of unsynchronized memory accesses between threads.
Two instructions are said to race if:
\begin{enumerate}
	\item they both access the same memory address,
	\item at least one is a write,
	\item the threads do not hold the same lock,
	\item and no synchronization enforces an order on the thread transitions (the {\em Happens-Before} relation).
\end{enumerate}
%In Figure~\ref{fig:example}, lines 3 and 5 each race with 2 and 6, and line 6 races with 8.
In Figure~\ref{fig:hb-example}, the pairs of lines marked with comments (A1/A2/B1/B2) race.

A data race analysis may be either {\em static} (inspecting source code) \cite{racerx} or {\em dynamic} (tracking individual accesses arising at run-time) \cite{tsan}.
This paper focuses exclusively on dynamic analysis,
so although our example refers to numbered source lines for ease of explanation,
in practice we are actually classifying the individual memory access events corresponding to those lines during execution.
Actually, each {\tt x++} statement likely compiles to two separate load and store instructions, so each of those two instructions from each of the two marked source lines pairwise will race (except for the two loads, which are both reads).

{\bf Variants of Happens-Before.}
        Most prior work focuses on {\em Happens-Before} \cite{lamport-clocks} as the order relation between accesses.
\cite{predictive-dr} and \cite{hybriddatarace} identify a problem with this approach:
it cannot identify access pairs separated by an unrelated lock operation which could race in an alternate interleaving,
as shown in the example program in Figure~\ref{fig:hb-example}(a).
We call such unreported access pairs {\em false negatives}.
\cite{hybriddatarace} introduces the {\em Limited Happens-Before} relation,
which will report such potential races
by considering only blocking operations like {\tt cond\_wait} to enforce the order.
However, consider the similar program in Figure~\ref{fig:hb-example}(b),
in which the access pair ceases to exist in the alternate interleaving.
Limited Happens-Before will report all potential races, avoiding false negatives \cite{tsan},
but at the cost of necessarily reporting some such {\em false positives}.

%Finally, the {\em Causally-Precedes} relation \cite{predictive-dr} %strikes a middle ground,
%extends Happens-Before to additionally report a subset of potential races while soundly avoiding false positives.
%It tracks conflicting accesses in intervening
%critical sections to determine whether lock events are unrelated to a potential race.
%Causally-Precedes will identify the potential race in Figure~\ref{fig:hb-example}(a), as the two critical sections do not conflict,
%although it can still miss true potential races in other cases.

In Chapter~\ref{chap:quicksand}, we use the Limited Happens-Before relation for our analysis.
The justification for this is that, while stand-alone data-race analyses must avoid inundating the user with false alarms \cite{racerx},
my work incorporates data-race analysis in an internal feedback loop, and reports only directly observed failures to the user.
Hence, I accept some overhead from false positives for the sake of more thorough testing.

%%%%%%%%%%%%%%%%%%%%%%%%%%%%%%%%%%%%%%%%%%%%%%%%%%%%%%%%%%%%%%%%%%%%%%%%%%%%%%%%

\section{Education}

In this thesis I will tackle Pebbles and Pintos, two different system architectures used in educational operating systems courses.
This section describes the projects which students implement and which Landslide tests.

\subsection{Pebbles}

In the course of a semester, students work on five programming assignments;
the first two are individual, and the remaining three are the products of two-person teams.
I will focus on the third and fourth of these, the thread library and kernel,
called ``P2'' and ``P3'' respectively (the project numbers start at 0).
The other three (a stack-crawling backtrace utility, a bare-metal game with device drivers, and a small extension to the P3 kernel) are not of concern in this thesis.
Both P2 and P3 are built using the {\em Pebbles} system call specification, outlined in Table~\ref{tab:syscalls}

\begin{table}
        \center
        \begin{tabular}{|l|p{0.75\textwidth}|}
                \hline
                \bf System call name & \bf Summary \\
                \hline
                \multicolumn{2}{c}{\em Lifecycle management} \\
                \hline
                \texttt{fork} & Duplicates the invoking task, including all memory regions. \\
                \texttt{thread\_fork} & Creates a new thread in the current task.\\
                \texttt{exec} & Replaces the program currently running in the invoking task with a new one specified. \\
                \texttt{set\_status} & Records the exit status of the current task. \\
                \texttt{vanish} & Terminates execution of the calling thread. \\
                \texttt{wait} & Blocks execution until another task terminates, and collects its exit status.\\
                \texttt{task\_vanish}* & Causes all threads of a task to \texttt{vanish}. \\
                \hline
                \multicolumn{2}{c}{\em Thread management} \\
                \hline
                \texttt{gettid} & Returns the ID of the invoking thread. \\
                \texttt{yield} & Defers execution to a specified thread. \\
                \texttt{deschedule} & Blocks execution of the invoking thread. \\
                \texttt{make\_runnable} & Wakes up another \texttt{deschedule}d thread. \\
                \texttt{get\_ticks} & Gets the number of timer ticks since bootup. \\
                \texttt{sleep} & Blocks a thread for a given number of ticks. \\
                \texttt{swexn} & Registers a user-space function as a software exception handler.\\
                \hline
                \multicolumn{2}{c}{\em Memory management} \\
                \hline
                \texttt{new\_pages} & Allocates a specified region of memory. \\
                \texttt{remove\_pages} & Deallocates same. \\
                \hline
                \multicolumn{2}{c}{\em Console I/O} \\
                \hline
                \texttt{getchar}* & Reads one character from keyboard input. \\
                \texttt{readline} & Reads the next line from keyboard input. \\
                \texttt{print} & Prints a given memory buffer to the console. \\
                \texttt{set\_term\_color} & Sets the color for future console output. \\
                \texttt{set\_cursor\_pos} & Sets the console cursor location. \\
                \texttt{get\_cursor\_pos} & Retrieves the console cursor location. \\
                \hline
                \multicolumn{2}{c}{\em Miscellaneous} \\
                \hline
                \texttt{ls} & Loads a given buffer with the names of files stored in the RAM disk ``file system.'' \\
                \texttt{halt} & Ceases execution of the operating system. \\
                \texttt{misbehave}* & Selects among several thread-scheduling policies. \\
                \hline
        \end{tabular}
        \caption{The Pebbles specifcation defines 25 system calls. Students are not required to implement ones marked with an asterisk (*), though the reference kernel provides them. }
        \label{tab:syscalls}
\end{table}

{\bf P2.}
% TODO

{\bf P3.}
% TODO

\subsection{Pintos}
% TODO
