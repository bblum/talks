\chapter{Education}
\label{chap:410}

This chapter proposes my plan to evaluate Landslide's effectiveness as a debugging aid for students in an educational setting.
This is the second of the three projects I am proposing for this thesis, and is currently ongoing work.

\section{Motivation}

In my MS thesis \cite{landslide}, I solicited students at CMU's Operating Systems Design and Implementation class
(henceforth, ``15-410'')
to volunteer at the end of the P3 project to annotate their kernels and try debugging them with Landslide.
However, the annotation burden undermined Landslide's purpose:
the only students willing to spend free time on manual instrumentation were biased to be those who were already doing well in the class,
and hence least likely to benefit from Landslide's debugging potential.
(Actually, even the best 15-410 students still have concurrency bugs,
but in principle, an educational tool must reach the more struggling students,
the so-called ``middle'' and/or ``bottom'' of the class.)
%
Requiring annotations hurts Landslide's case as a grading tool, as well:
TAs need to understand the kernel to begin with in order to annotate correctly,
and while achieving such understanding they may as well grade it by hand, as before.

Since then, I've extended Landslide to support testing P2 thread libraries (Section~\ref{sec:pebbles}) as well.
Because P2 mandates specific function names for the project's internal APIs
-- most importantly, for the concurrency primitives --
Landslide can automatically annotate arbitrary student implementations with no manual effort required of the user (whether student or TA).
The addition of Quicksand and Iterative Deepening (Chapter~\ref{chap:quicksand}) partly fulfills this purpose,
freeing student attention from the issue of which state spaces to test.
This chapter will detail my further techniques and evaluation which are specific to educational use.

\section{Implementation Details}

Every stateless model checker must make some assumptions about the tested programs' concurrency model \cite{chess}.
However, arbitrary programs may break conventional disciplines of concurrent programs, while still being bug-free.
For example, thread communication via ad-hoc {\tt yield} loops may appear as an infinite loop, livelock, or deadlock.
This is especially true of student code,
written by people who are just learning concurrent programming discipline for the first time,
and/or written under the time pressure of a project deadline.
To be an effective tool for struggling students,
a model checker must somehow coax adversarial programs to fit its concurrency model, to effectively test for real bugs,
rather than rejecting them outright on some stylistic or disciplinary grounds.

Fully-automatic instrumentation of student P2s has been no walk in the park.
I have equipped Landslide with several powerful algorithms and heuristics for handling the most common anti-patterns in student submissions.

\begin{itemize}
	\item {\bf Yield-blocking.}
		Landslide recognizes open-coded busy-wait loops used for ad-hoc synchronization,
		and is able to treat threads as blocked (or ``disabled'' in model-checking parlance),
		avoiding getting stuck in an infinitely-long interleaving (or ``cyclic state space'')
		which should never arise during normal execution.
		The heuristically-driven algorithm is as follows:
		\begin{itemize}
			\llitem Whenever a thread performs a {\tt yield}, {\tt xchg}, or other atomic instruction,
				Landslide increments a per-thread counter to track its (supposed) busy-wait loop iterations.
				A thread's counter is reset any time it performs some ``interesting'' activity not likely to appear in a true busy-wait loop:
				any condvar, semaphore, or rwlock invocation (but not mutexes),
				and the beginning or end of any thread library function (create, join, or exit).
				\footnote{Implemented in {\tt check\_user\_yield\_activity} and {\tt check\_user\_xchg} in {\tt user\_sync.c}.}
				% TODO: Wow, what happened to KEEP_RUNNING_YIELDING_THREADS?? I swear I had some logic in arbiter.c about that.
			\llitem When a thread's loop counter reaches some heuristic limit (10 for {\tt yield}s, 100 for {\tt xchg}s),
				% Or 20 for xchgs which end up being data-race PPs.
				Landslide marks the thread blocked (or ``disabled'', in model-checking parlance),
				just as though it had invoked {\tt deschedule}.
				It also retroactively disables that thread at all preceding {\tt yield}s/{\tt xchg}s in that sequence,
				which prevents DPOR from trying to use each as a preemption point,
				and avoids a state space explosion (by a factor of the heuristic yield limit).
				% Well, only the whole state space will explode if this yield-block *always* happens.
				% Otherwise it's (heuristic limit) * (size of subtree) * (number of yield-block occurrences).
				% Tbf, I guess it's very likely for that to be half, or more, of the whole space.
				\footnote{Implemented in {\tt update\_blocked\_transition} in {\tt user\_sync.c}.}
			\llitem When retroactively disabling a thread across all its preceding loop iterations,
				Landslide's state space estimator must account for the ``pruning'' of duplicate subtrees at those (now disabled) preemption points.
				% TODO: Yeah, thinking about this again, I'm pretty sure untag_blocked_branch is deadcode.
				If in any previous thread interleaving, DPOR tagged the now-yield-blocked thread to interleave at another thread's preemption point,
				the estimator would have included that potential subtree in its computation of how much unexplored state space exists.
				Accordingly, in this case Landslide will invert the estimation algorithm,
				including propagating the reduced subtree size all the way to the state space's root.
				\footnote{Implemented in {\tt untag\_blocked\_branch} in {\tt estimate.c}.}
			\llitem Landslide can precisely identify when another thread may trigger the yield-blocking one to fall out of its loop,
				by analyzing the shared memory conflicts involving only accesses performed in the loop.
				At any such memory conflict, Landslide will reenable the yielding thread.
				(If the other thread's access does not fulfill the yielding thread's wait condition,
				the latter will just re-trigger the heuristic and become blocked again.)
				% Hmm.... if two threads are in a deadlock between two ad-hoc loops,
				% and each has some spurious access which kicks the other,
				% it will show up as an infinite loop instead. I guess that's ok?
				\footnote{Implemented in {\tt check\_unblock\_yield\_loop} in {\tt user\_sync.c}.}
		\end{itemize}
		This approach is similar to the Fair-Bounded Search algorithm in \cite{bpor},
		although it avoids a major assumption of the latter (threads {\tt yield} if and only if not making progress),
		and also avoids the need to iteratively deepen the yield bound by fixing it as a heuristic constant.
		% Actually, I don't think that's true. Using DPOR to unblock a yield-looping thread at any conflict covers this.
		%The cost of this heuristic is that falsely identifying threads as blocked can lead to unsoundness (...)
	\item {\bf False-positive deadlock avoidance.}
		In contrast to its treatment of data races, Landslide must never report a false-positive {\em bug}.
		If its heuristics falsely identify a thread as blocked, and all other threads are truly blocked,
		waiting on some progress from that thread, Landslide would report a deadlock bug,
		and confuse students horribly.

		The yield-loop heuristic assumes that ``too many'' yields or atomics should not arise during normal, non-looping execution of thread library routines.
		Though extremely rare, this assumption can be violated by an adversarial student submission.
		More often, Landslide can falsely block threads in the special case of {\tt mutex\_test} (see \cite{quicksand}),
		where it uses preemption points within the implementation of {\tt mutex\_lock} itself.
		False deadlocks can also arise from the heuristic blocking of ICB \cite{chess-icb}
		(used in Quicksand's control experiments, Section~\ref{sec:quicksand-eval}).

		When a deadlock arises under conditions where one or more threads are heuristically blocked,
		Landslide attempts to refute it as a false positive by artificially unblocking all heuristically-blocked threads%
		\footnote{If any threads are ICB-blocked, I prioritize waking those before trying to wake any yield-blocked threads.
		% Hmm... I don't remember exactly why.
		Waking all threads at once here can lead to unsoundness.}
		If the deadlock is true, each thread will immediately trigger the yield-blocking heuristic again,
		bringing the system back into deadlocked state.
		Landslide then repeats this process a heuristic constant number of times (128),
		allowing the program that many chances to make progress before proclaiming deadlock.
		(Note that this heuristic cannot miss true deadlocks as false negatives.)
		\footnote{Implemented in {\tt try\_avoid\_fp\_deadlock} in {\tt arbiter.c}.}
	\item {\bf Lock hand-off.}
		A common, though discouraged, idiom for implementing thread destruction involves one thread ``handing off'' ownership of a mutex to another.
		That thread will then release the lock with no corresponding acquire in its own execution.
		% In this case, some other thread communication enforces an ordering ... hmm,, pure vs limited hb?
		Although Landslide cannot easily recognize at what point the latter thread's accesses are protected by that lock for the sake of data-race analysis,
		potentially leading to false positive races,
		it must release the lock in its bookkeeping to avoid false {\em negatives} if any later access that should be protected by that lock isn't.
		When releasing a lock, if absent from the current thread's lockset, Landslide searches the locksets of all other existing threads, and releases it there.
		Landslide can instead optionally be configured to treat lock hand-off as an outright bug, as a matter of discipline.
		\footnote{Implemented in {\tt lockset\_remove} in {\tt lockset.c}.}

	% Not really research.
	%\item Other small engineering fixes, such as: recognizing when a system call's access to user memory constitutes a ``communication back-channel'' for user threads that should be considered for DPOR;
\end{itemize}

However, some of the less common offenses are both more difficult to handle algorithmically, and also more worrying from a pedagogical point of view.
%After some collaboration with David Eckhardt (15-410 professor and member of this thesis committee),
For the following cases, we configured Landslide to abort, and warn the student that they must find a better solution before it could test their code.

\begin{itemize}
	\item Busy-wait loops containing neither {\tt yield} nor {\tt xchg} (nor any other atomic instruction), such as {\tt while (!other\_thread\_ready) continue;}.
		This blurs the line between anti-pattern and concurrency bug:
		because it does not yield the CPU, a uni-processor machine must wait for the next timer tick (several milliseconds!) before making any progress;
		also, because it does not use atomic instructions, an optimizing compiler may reorder or even delete the loop's accesses.
		%
		Landslide also cannot easily identify it as similar to message-passing,
		appearing indistinguishable
		% not really? there would be dpor conflix, at least
		from a thread-local infinite computation,
		which is of course impossible to judge for halting \cite{entscheidungsproblem}.

		In such a case, Landslide will issue a bug report with the special message:
		{\em I have run a loop in [function name] an alarming number of times.
		This version of Landslide cannot distinguish between this loop being infinite versus merely undesirable.
		Please refer to the ``Synchronization (2)'' lecture.}
	\item Recursive mutex use (i.e., locking the same mutex twice in the same thread, then subsequently unlocking it twice).
		While it would not be difficult for Landslide's lock-sets to support recursive locking (using a nesting counter instead of a boolean flag),
		that would assume the corresponding mutex implementation provides the same semantics,
		which is risky business with student code. % ;)
		Furthermore, recursive locking is not an obvious solution to any of P2's challenges;
		far more often, it arises when a student's mutex tries to {\tt malloc} some internal state,
		which itself requires a mutex for safe allocation, which can lead to stack overflow and a crash.

		In such a case, Landslide will issue a bug report with the special message:
		{\em This version of Landslide cannot debug recursive implementations of mutex\_lock.
		Please examine this stack trace and determine for yourself whether it indicates a bug.}
\end{itemize}

\section{Landslide as a Debugging Tool}
\label{sec:studence}

TODO

\section{Landslide as a Grading Tool}
\label{sec:grading}

TODO
