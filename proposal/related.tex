\chapter{Related Work}
\label{chap:related}

In this chapter I provide a brief tour of the related work.

\section{History of Stateless Model Checking}

{\bf Hall of Fame.}
Stateless model checking dates back to Verisoft \cite{verisoft}, the 1997 tool which first attempted to exhaustively explore the possibile ways to interleave threads.
Since then, researchers have built many tools along the same lines to test many kinds of programs.
%list of mcs follows
The most popular model checker, according to citation count, is Microsoft Research's CHESS \cite{chess},
a checker for userspace C++ programs which preempts at synchronization APIs.
%which introduced the Iterative Context Bounding technique \cite{chess-icb} for heuristically ordering the search to uncover bugs faster.
%
Other checkers include MaceMC \cite{macemc}, MoDist \cite{modist}, SAMC \cite{samc}, ETA \cite{dbug-retreat},
each of which limit thread communication to the boundaries of a message-passing API;
%
dBug \cite{dbug-ssv}, another CMU original, similar to CHESS;
%
and finally SPIN \cite{spin} and Inspect \cite{inspect} which can preempt at any shared variable access.
%
Perhaps by now, Landslide itself deserves a spot on this list: I first introduced it in my MS thesis \cite{landslide},
and it is most notable for testing kernel-level code through the use of a simulator.

{\bf Search Strategies}
To date a number of techniques have been proposed to mitigate exponential explosion, the Sisyphean rock of stateless model checkers.
Dynamic Partial Order Reduction (DPOR) \cite{dpor}, the baseline approach, prunes redundant interleavings by identifying independent thread transitions which can commute without changing the program state.
It is a {\em sound} reduction algorithm, meaning it will never fail to test a possible program behavior, despite skipping many execution sequences.
DPOR has since been extended in several ways:
a provably optimal version which guarantees to explore no more than one interleaving from each equivalence class \cite{optimal-dpor},
% TODO: a 3rd example?
and an extension of the algorithm to include nondeterminism arising from weak memory models \cite{tsopso}.
Recently, SATCheck and Maximal Causality Reduction have emerged as better-performing alternatives to DPOR.
These algorithms use SMT solvers \cite{z3} to identify additional equivalences by analyzing the values read and written by each memory access.

{\bf Other theoretical advances.}
A number of other techniques address the problem that even with DPOR, a state space may be too large.
Iterative Context Bounding (ICB) \cite{chess-icb} is a search ordering heuristic which prioritizes interleavings with fewer preemptions,
according to the insight that most bugs require fewer preemptions to expose.
Bounded Partial Order Reduction \cite{bpor} adapts DPOR to be soundly compatible with ICB.
Preemption Sealing \cite{sealing} allows programmers to exclude preemption points arising from trusted source code modules,
reducing the state space by limiting the scope of the test.
Probabilistic Concurrency Testing \cite{randomized-scheduler} eschews DPOR's depth-first approach entirely (as well as any potential for completing the state space), randomly sampling a broad cross-section of the state space and providing probabilistic bounds on uncovering bugs.

\section{Educational Testing Tools}

\section{Real-World Techniques}
