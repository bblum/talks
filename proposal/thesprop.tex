%for a more compact document, add the option openany to avoid
%starting all chapters on odd numbered pages
\documentclass[12pt]{cmuthesis}

% This is a template for a CMU thesis.  It is 18 pages without any content :-)
% The source for this is pulled from a variety of sources and people.
% Here's a partial list of people who may or may have not contributed:
%
%        bnoble   = Brian Noble
%        caruana  = Rich Caruana
%        colohan  = Chris Colohan
%        jab      = Justin Boyan
%        josullvn = Joseph O'Sullivan
%        jrs      = Jonathan Shewchuk
%        kosak    = Corey Kosak
%        mjz      = Matt Zekauskas (mattz@cs)
%        pdinda   = Peter Dinda
%        pfr      = Patrick Riley
%        dkoes = David Koes (me)

% My main contribution is putting everything into a single class files and small
% template since I prefer this to some complicated sprawling directory tree with
% makefiles.

% some useful packages
\usepackage{times}
\usepackage{fullpage}
\usepackage{graphicx}
\usepackage{amsmath}
\usepackage[charter]{mathdesign}
\usepackage[numbers,sort]{natbib}
\usepackage[backref,pageanchor=true,plainpages=false, pdfpagelabels, bookmarks,bookmarksnumbered,
%pdfborder=0 0 0,  %removes outlines around hyper links in online display
]{hyperref}
\usepackage{subfigure}

% Approximately 1" margins, more space on binding side
%\usepackage[letterpaper,twoside,vscale=.8,hscale=.75,nomarginpar]{geometry}
%for general printing (not binding)
\usepackage[letterpaper,twoside,vscale=.8,hscale=.75,nomarginpar,hmarginratio=1:1]{geometry}

% Provides a draft mark at the top of the document. 
\draftstamp{\today}{DRAFT}

\begin {document} 
\frontmatter

%initialize page style, so contents come out right (see bot) -mjz
\pagestyle{empty}

\title{ %% {\it \huge Thesis Proposal}\\
{\bf Practical Concurrency Testing}}
\author{Ben Blum}
\date{Janucember 2016}
\Year{2016}
\trnumber{}

\committee{
Garth Gibson, Chair \\
David A. Eckhardt \\
Yet another person \\
Someone from a strange and faraway land
}

\support{}
\disclaimer{}

% copyright notice generated automatically from Year and author.
% permission added if \permission{} given.

\keywords{landslide terminal, baggage claim, public transportation, ticketing}

\maketitle

\begin{dedication}
	For my dog, Louie.
\end{dedication}

\pagestyle{plain} % for toc, was empty

%% Obviously, it's probably a good idea to break the various sections of your thesis
%% into different files and input them into this file...

\begin{abstract}
Concurrent programming presents a challenge to students and experts alike because of the complexity of multithreaded interactions and the difficulty to reproduce and reason about bugs.
Stateless model checking is a concurrency testing approach which forces a program to interleave its threads in many different ways, checking for bugs each time.
This technique is powerful, in principle capable of finding any nondeterministic bug in finite time,
but suffers from exponential explosion as program size increases.
Checking an exponential number of thread interleavings is not a practical or predictable approach for programmers to find concurrency bugs before their project deadlines.

In this thesis, I propose to make stateless model checking more practical for human use by way of several new techniques.
I have built Landslide, a stateless model checker specializing in student projects for undergraduate operating systems classes.
Landslide includes a novel algorithm for automatically managing
%the exploration of
multiple state spaces according to their estimated completion times,
which I will show quickly finds bugs should they exist and also quickly verifies correctness otherwise.
I will evaluate Landslide's suitability for inexpert use by presenting the results of many semesters providing it to students in 15-410, CMU's Operating System Design and Implementation class.
%Finally, I will present several new techniques that allow stateless model checking to be practically employed on real-world programs.
Finally, I will explore broader impact by extending Landslide to test real-world programs and to be used by students at other universities.
\end{abstract}

\begin{acknowledgments}
My advisor is cool.

But I am cooler.
\end{acknowledgments}



\tableofcontents
%\listoffigures
%\listoftables

\mainmatter

%% Double space document for easy review:
%\renewcommand{\baselinestretch}{1.66}\normalsize

% The other requirements Catherine has:
%
%  - avoid large margins.  She wants the thesis to use fewer pages, 
%    especially if it requires colour printing.
%
%  - The thesis should be formatted for double-sided printing.  This
%    means that all chapters, acknowledgements, table of contents, etc.
%    should start on odd numbered (right facing) pages.
%
%  - You need to use the department standard tech report title page.  I
%    have tried to ensure that the title page here conforms to this
%    standard.
%
%  - Use a nice serif font, such as Times Roman.  Sans serif looks bad.
%
% Other than that, just make it look good...


\chapter{Introduction}

Modern computer architectures have turned to increasing CPU core count, rather than clock speed, to improve processing power.
%To take advantage of multiple cores, programs must be written {\em concurrently}
To take advantage of multiple cores for performance, programmers must write software to execute {\em concurrently} --
using multiple {\em threads} which execute multiple parts of a program's logic simultaneously.
However, when threads access the same shared data, they may interleave in unexpected ways which change the outcomes of their execution.
When an unexpected interleaving produces undesirable program behaviour,
for example, by corrupting shared data structures,
we call it a {\em concurrency bug}.
Concurrency bugs are notoriously hard for programmers to find and debug
because the specific thread interleaving required to trigger them arises at random during normal execution,
and often with very low probability.
%Concurrency bugs are notoriously hard to find and reproduce because they only appear in specific thread interleavings, which arise at random during normal program execution.
% TODO(LAYMAN): give example of trying to open car door at same time as friend turns key to unlock it.

Most commonly, a programmer searches for concurrency bugs in her code by running it many times (in parallel, in serial, or both),
hoping that eventually, it will run according to the particular interleaving required to expose a hypothetical bug.
This technique, known as {\em stress testing}, is unreliable,
providing no guarantee of finding the failing interleaving in any finite amount of time, even when a bug does exist.
It also provides no assurance of correctness:
when finished, there is no way of knowing how much of the space of possible thread interleavings was actually tested.
%it may by chance test only a single interleaving over and over again!
Nevertheless, stress testing remains popular because of how easily a programmer can use it:
she simply wraps her program in a loop, sets it to run overnight, and kills it if her patience runs out before it finds a bug.

{\em Stateless model checking} \cite{verisoft} is an alternative way to test for concurrency bugs,
or to verify their absence,
which provides more reliable coverage, progress, and verification than stress testing.
A stateless model checker tests a program by forcing it to execute a new unique thread interleaving on each iteration of the test,
capturing and controlling the randomness in a finite state space of all possible interleavings.

Unfortunately, the size of these state spaces is exponentially proportional to the size of the tested program.
% TODO(LAYMAN): explain exponential explosion by relating the parable of grains of rice on a chessboard.
For even moderately-sized programs, there may be more possible ways to interleave every thread's every instruction
than particles in the universe.
Accordingly, a programmer who wants her test to make reasonable progress through the state space must choose a subset of ways that her threads could interleave,
focusing on fully testing that subset, while ignoring other possibilities she doesn't care about.
%However, choosing this subset is a difficult trade-off for humans to make before even knowing whether a bug exists.
However, it is difficult to choose a subset of thread interleavings that will produce a meaningful, yet feasible test.
Until computers can automatically navigate this trade-off in some intelligent way,
programmers will continue to fall back to the random approach of stress testing.

Another problem stateless model checking suffers is that certain types of programs cannot be tested without the programmer putting forth some manual instrumentation effort.
For example, operating system kernels implement their own sources of concurrency and their own synchronization primitives,
so the checker needs to be told how to identify and control the execution of each thread.
Some expert programmers may be willing to add manual annotations to their code,
if they are already veteran concurrency research wizards,
but required manual effort is a serious downside for anyone with a looming deadline,
and especially so for students who are still learning basic concurrency principles.
%We should not expect programmers to add effortful manual annotations to their code,
%or they will abandon our fancy technique to instead simply run stress tests until their deadline tomorrow evening.

This thesis will solve both problems discussed above.
My thesis statement is as follows:

\begin{center}
	{\em my research is cool. please let me graduate.}
\end{center}

\vspace{1em}

I have built Landslide \cite{landslide}, a stateless model checker for thread libraries and kernels,
and I have developed some techniques for automatically choosing the best thread interleavings to test
and for automatically instrumenting operating system kernels in an educational setting.
This thesis will comprise three major contributions:

\begin{itemize}
	\item {\bf Meaningful state spaces (\ref{chap:pps}).}
		I will present {\em Iterative Deepening}, a new algorithm for navigating the trade-off in how many preemption points to test at once.
		Iterative Deepening incorporates state space estimation \cite{estimation} to decide on-the-fly whether each state space is worth pursuing, and uses data race analysis \cite{tsan} to find new preemption point candidates based on a program's dynamic behaviour.
		This section will include a large evaluation of the technique, comparing its performance to three prior work approaches across 629 unique tests.
		I will show that Iterative Deepening of preemption points outperforms prior work in terms both of finding bugs quickly and of completely verifying correctness when no bug exists.
	\item {\bf Educational use (\ref{chap:410}).}
		For the past three semesters, I have offered a fully-automated version of Landslide to students in 15-410, CMU's undergraduate Operating System Design and Implementation class \cite{kspec,thrlib}, for use as a debugging aid during the thread library project.
		I will continue these user studies, and use the data to evaluate the suitability of stateless model checking in an educational setting.
		%, investigating the following questions in particular:
		%\begin{itemize}
		%	\item Does Landslide improve project submission quality when used as a debugging aid?
		%	\item Does Landslide teach students lessons in concurrent programming 
		%\end{itemize}
	\item {\bf Broader impact (\ref{chap:lipservice}).}:
		Landslide includes several new techniques for testing kernels and thread libraries with little or no manual instrumentation required.
		So far, a fully-automatic testing mode is available only for 15-410 thread library projects.
		To prove these techniques are relevant beyond CMU's walls, I will extend Landslide to handle both Pintos kernel projects from other universities \cite{pintos} and ``real-world'' Linux programs.
		I will conduct a user study in which students at another university test their Pintoses with Landslide,
		and evaluate Landslide's ability to find known and/or new bugs in real-world programs.
\end{itemize}

% TODO(thesis): Other chapters of this thesis include...


\chapter{Meaningful State Spaces}
\label{chap:pps}

\chapter{Education}
\label{chap:410}

\chapter{Broader Impact}
\label{chap:lipservice}




%Reduction
%techniques such as Dynamic Partial Order Reduction \cite{dpor} and Maximal Causality Reduction \cite{mcr} expand the limits of feasible test completion,
%and search ordering strategies such as Iterative Context Bounding \cite{chess-icb} \revision{encourage}~bugs to be found sooner in a given space should they exist.


%Topics of thesis:
%1. Data race preemption points / iterative deepening
%2. Education in 410. Evaluation questions are:
%	2a. Does P2 submission quality improve?
%	2b. Does P3 submission quality improve correlated with students who used (got the most out of) Landslide during P2?
%	2c. Does Landslide reach the bottom of the class? (Look at bottom-of-class P0s/P1s, see how they improve)
%3. Broader impact
%	3a. Real world programs (and the challenges therein?)
%	3b. Give to pintos kids (and open-source landslide on bochs in doing so)

\chapter{Conclusion}

%\appendix
%\include{appendix}

\backmatter

%\renewcommand{\baselinestretch}{1.0}\normalsize

% By default \bibsection is \chapter*, but we really want this to show
% up in the table of contents and pdf bookmarks.
\renewcommand{\bibsection}{\chapter{\bibname}}
%\newcommand{\bibpreamble}{This text goes between the ``Bibliography''
%  header and the actual list of references}
\bibliographystyle{plainnat}
\bibliography{citations} %your bib file

\end{document}
