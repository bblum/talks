%for a more compact document, add the option openany to avoid
%starting all chapters on odd numbered pages
\documentclass[12pt]{cmuthesis}

% This is a template for a CMU thesis.  It is 18 pages without any content :-)
% The source for this is pulled from a variety of sources and people.
% Here's a partial list of people who may or may have not contributed:
%
%        bnoble   = Brian Noble
%        caruana  = Rich Caruana
%        colohan  = Chris Colohan
%        jab      = Justin Boyan
%        josullvn = Joseph O'Sullivan
%        jrs      = Jonathan Shewchuk
%        kosak    = Corey Kosak
%        mjz      = Matt Zekauskas (mattz@cs)
%        pdinda   = Peter Dinda
%        pfr      = Patrick Riley
%        dkoes = David Koes (me)

% My main contribution is putting everything into a single class files and small
% template since I prefer this to some complicated sprawling directory tree with
% makefiles.

% some useful packages
\usepackage{times}
\usepackage{fullpage}
\usepackage{graphicx}
\usepackage{amsmath}
\usepackage[charter]{mathdesign}
\usepackage[numbers,sort]{natbib}
\usepackage[backref,pageanchor=true,plainpages=false, pdfpagelabels, bookmarks,bookmarksnumbered,
%pdfborder=0 0 0,  %removes outlines around hyper links in online display
]{hyperref}
\usepackage{subfigure}

% Approximately 1" margins, more space on binding side
%\usepackage[letterpaper,twoside,vscale=.8,hscale=.75,nomarginpar]{geometry}
%for general printing (not binding)
\usepackage[letterpaper,twoside,vscale=.8,hscale=.75,nomarginpar,hmarginratio=1:1]{geometry}

% Provides a draft mark at the top of the document. 
\draftstamp{\today}{DRAFT}

\begin {document} 
\frontmatter

%initialize page style, so contents come out right (see bot) -mjz
\pagestyle{empty}

\title{ %% {\it \huge Thesis Proposal}\\
{\bf Practical Concurrency Testing}}
\author{Ben Blum}
\date{Janucember 2016}
\Year{2016}
\trnumber{}

\committee{
Garth Gibson, Chair \\
David A. Eckhardt \\
Yet another person \\
Someone from a strange and faraway land
}

\support{}
\disclaimer{}

% copyright notice generated automatically from Year and author.
% permission added if \permission{} given.

\keywords{landslide terminal, baggage claim, public transportation, ticketing}

\maketitle

\begin{dedication}
	For my dog, Louie.
\end{dedication}

\pagestyle{plain} % for toc, was empty

%% Obviously, it's probably a good idea to break the various sections of your thesis
%% into different files and input them into this file...

\begin{abstract}
Concurrent programming presents a challenge to students and experts alike because of the complexity of multithreaded interactions and the difficulty to reproduce and reason about bugs.
Stateless model checking is a concurrency testing approach which forces a program to interleave its threads in many different ways, checking for bugs each time.
This technique is powerful, in principle capable of finding any nondeterministic bug in finite time,
but suffers from exponential explosion as program size increases.
Checking an exponential number of thread interleavings is not a practical or predictable approach for programmers to find concurrency bugs before their project deadlines.

In this thesis, I propose to make stateless model checking more practical for human use by way of several new techniques.
I have built Landslide, a stateless model checker for testing student projects for undergraduate operating systems classes.
Landslide includes a novel algorithm for automatically managing
%the exploration of
multiple state spaces according to their estimated completion times,
which I will show quickly finds bugs should they exist and also quickly verifies correctness otherwise.
I will evaluate Landslide's suitability for inexpert use by presenting the results of a many-semester study giving it as a debugging aid to students in 15-410, CMU's Operating System Design and Implementation class.
Finally, I will present several new techniques that allow stateless model checking to be practically employed on real-world programs.
\end{abstract}

\begin{acknowledgments}
My advisor is cool.

But I am cooler.
\end{acknowledgments}



\tableofcontents
%\listoffigures
%\listoftables

\mainmatter

%% Double space document for easy review:
%\renewcommand{\baselinestretch}{1.66}\normalsize

% The other requirements Catherine has:
%
%  - avoid large margins.  She wants the thesis to use fewer pages, 
%    especially if it requires colour printing.
%
%  - The thesis should be formatted for double-sided printing.  This
%    means that all chapters, acknowledgements, table of contents, etc.
%    should start on odd numbered (right facing) pages.
%
%  - You need to use the department standard tech report title page.  I
%    have tried to ensure that the title page here conforms to this
%    standard.
%
%  - Use a nice serif font, such as Times Roman.  Sans serif looks bad.
%
% Other than that, just make it look good...


\chapter{Introduction}
\chapter{Conclusion}

%\appendix
%\include{appendix}

\backmatter

%\renewcommand{\baselinestretch}{1.0}\normalsize

% By default \bibsection is \chapter*, but we really want this to show
% up in the table of contents and pdf bookmarks.
\renewcommand{\bibsection}{\chapter{\bibname}}
%\newcommand{\bibpreamble}{This text goes between the ``Bibliography''
%  header and the actual list of references}
\bibliographystyle{plainnat}
\bibliography{register} %your bib file

\end{document}
